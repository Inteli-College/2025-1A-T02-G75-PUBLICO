% Options for packages loaded elsewhere
\PassOptionsToPackage{unicode}{hyperref}
\PassOptionsToPackage{hyphens}{url}
%
\documentclass[
]{article}
\usepackage{amsmath,amssymb}
\usepackage{lmodern}
\usepackage{iftex}
\usepackage{float}
\usepackage{graphicx} 
\usepackage[backend=biber,style=apa]{biblatex}
\addbibresource{sample.bib}
\ifPDFTeX
  \usepackage[T1]{fontenc}
  \usepackage[utf8]{inputenc}
  \usepackage{textcomp} % provide euro and other symbols
\else % if luatex or xetex
  \usepackage{unicode-math}
  \defaultfontfeatures{Scale=MatchLowercase}
  \defaultfontfeatures[\rmfamily]{Ligatures=TeX,Scale=1}
\fi
% Use upquote if available, for straight quotes in verbatim environments
\IfFileExists{upquote.sty}{\usepackage{upquote}}{}
\IfFileExists{microtype.sty}{% use microtype if available
  \usepackage[]{microtype}
  \UseMicrotypeSet[protrusion]{basicmath} % disable protrusion for tt fonts
}{}
\makeatletter
\@ifundefined{KOMAClassName}{% if non-KOMA class
  \IfFileExists{parskip.sty}{%
    \usepackage{parskip}
  }{% else
    \setlength{\parindent}{0pt}
    \setlength{\parskip}{6pt plus 2pt minus 1pt}}
}{% if KOMA class
  \KOMAoptions{parskip=half}}
\makeatother
\usepackage{xcolor}
\usepackage{color}
\usepackage{fancyvrb}
\newcommand{\VerbBar}{|}
\newcommand{\VERB}{\Verb[commandchars=\\\{\}]}
\DefineVerbatimEnvironment{Highlighting}{Verbatim}{commandchars=\\\{\}}
% Add ',fontsize=\small' for more characters per line
\newenvironment{Shaded}{}{}
\newcommand{\AlertTok}[1]{\textcolor[rgb]{1.00,0.00,0.00}{\textbf{#1}}}
\newcommand{\AnnotationTok}[1]{\textcolor[rgb]{0.38,0.63,0.69}{\textbf{\textit{#1}}}}
\newcommand{\AttributeTok}[1]{\textcolor[rgb]{0.49,0.56,0.16}{#1}}
\newcommand{\BaseNTok}[1]{\textcolor[rgb]{0.25,0.63,0.44}{#1}}
\newcommand{\BuiltInTok}[1]{\textcolor[rgb]{0.00,0.50,0.00}{#1}}
\newcommand{\CharTok}[1]{\textcolor[rgb]{0.25,0.44,0.63}{#1}}
\newcommand{\CommentTok}[1]{\textcolor[rgb]{0.38,0.63,0.69}{\textit{#1}}}
\newcommand{\CommentVarTok}[1]{\textcolor[rgb]{0.38,0.63,0.69}{\textbf{\textit{#1}}}}
\newcommand{\ConstantTok}[1]{\textcolor[rgb]{0.53,0.00,0.00}{#1}}
\newcommand{\ControlFlowTok}[1]{\textcolor[rgb]{0.00,0.44,0.13}{\textbf{#1}}}
\newcommand{\DataTypeTok}[1]{\textcolor[rgb]{0.56,0.13,0.00}{#1}}
\newcommand{\DecValTok}[1]{\textcolor[rgb]{0.25,0.63,0.44}{#1}}
\newcommand{\DocumentationTok}[1]{\textcolor[rgb]{0.73,0.13,0.13}{\textit{#1}}}
\newcommand{\ErrorTok}[1]{\textcolor[rgb]{1.00,0.00,0.00}{\textbf{#1}}}
\newcommand{\ExtensionTok}[1]{#1}
\newcommand{\FloatTok}[1]{\textcolor[rgb]{0.25,0.63,0.44}{#1}}
\newcommand{\FunctionTok}[1]{\textcolor[rgb]{0.02,0.16,0.49}{#1}}
\newcommand{\ImportTok}[1]{\textcolor[rgb]{0.00,0.50,0.00}{\textbf{#1}}}
\newcommand{\InformationTok}[1]{\textcolor[rgb]{0.38,0.63,0.69}{\textbf{\textit{#1}}}}
\newcommand{\KeywordTok}[1]{\textcolor[rgb]{0.00,0.44,0.13}{\textbf{#1}}}
\newcommand{\NormalTok}[1]{#1}
\newcommand{\OperatorTok}[1]{\textcolor[rgb]{0.40,0.40,0.40}{#1}}
\newcommand{\OtherTok}[1]{\textcolor[rgb]{0.00,0.44,0.13}{#1}}
\newcommand{\PreprocessorTok}[1]{\textcolor[rgb]{0.74,0.48,0.00}{#1}}
\newcommand{\RegionMarkerTok}[1]{#1}
\newcommand{\SpecialCharTok}[1]{\textcolor[rgb]{0.25,0.44,0.63}{#1}}
\newcommand{\SpecialStringTok}[1]{\textcolor[rgb]{0.73,0.40,0.53}{#1}}
\newcommand{\StringTok}[1]{\textcolor[rgb]{0.25,0.44,0.63}{#1}}
\newcommand{\VariableTok}[1]{\textcolor[rgb]{0.10,0.09,0.49}{#1}}
\newcommand{\VerbatimStringTok}[1]{\textcolor[rgb]{0.25,0.44,0.63}{#1}}
\newcommand{\WarningTok}[1]{\textcolor[rgb]{0.38,0.63,0.69}{\textbf{\textit{#1}}}}
\usepackage{longtable,booktabs,array}
\usepackage{calc} % for calculating minipage widths
% Correct order of tables after \paragraph or \subparagraph
\usepackage{etoolbox}
\makeatletter
\patchcmd\longtable{\par}{\if@noskipsec\mbox{}\fi\par}{}{}
\makeatother
% Allow footnotes in longtable head/foot
\IfFileExists{footnotehyper.sty}{\usepackage{footnotehyper}}{\usepackage{footnote}}
\makesavenoteenv{longtable}
\setlength{\emergencystretch}{3em} % prevent overfull lines
\providecommand{\tightlist}{%
  \setlength{\itemsep}{0pt}\setlength{\parskip}{0pt}}
\setcounter{secnumdepth}{-\maxdimen} % remove section numbering
\ifLuaTeX
  \usepackage{selnolig}  % disable illegal ligatures
\fi
\IfFileExists{bookmark.sty}{\usepackage{bookmark}}{\usepackage{hyperref}}
\IfFileExists{xurl.sty}{\usepackage{xurl}}{} % add URL line breaks if available
\urlstyle{same} % disable monospaced font for URLs
\hypersetup{
  hidelinks,
  pdfcreator={LaTeX via pandoc}}

\author{}
\date{}

\begin{document}

\tableofcontents
\newpage

\hypertarget{introduction}{%
\section{Introduction}\label{market-overview}}
In recent years, the convergence of decentralized finance (DeFi) and artificial intelligence (AI) has given rise to a new breed of tools designed to analyze, interpret, and act on blockchain data in real time. Among these, SWQuery emerges as a robust and developer-friendly solution within the DefAI ecosystem, providing simplified access to complex data across the Solana blockchain. By abstracting away low-level infrastructure concerns and integrating advanced querying, risk analysis, and live feed capabilities, SWQuery enables developers, analysts, and investors to make informed decisions with confidence.

This whitepaper presents a comprehensive overview of the context, design, and technical considerations behind SWQuery. From market positioning to architectural design, functional requirements, and testing frameworks, each section builds upon the last to provide a full understanding of the system’s value proposition and engineering backbone.

To ground SWQuery within its broader context, we begin with an analysis of the current landscape. The Market Overview highlights the growing ecosystem of DefAI tools and the positioning of SWQuery among similar projects that attempt to bridge decentralized data with intelligent analytics.

\hypertarget{market-overview}{%
\section{Market Overview}\label{market-overview}}

\begin{figure}[H]
\centering
\includegraphics[width=1\linewidth]{benchmark.png}
\caption{\label{fig:benchmark}This image represents the current market overview.}
\end{figure}

\hypertarget{research-analyst-agent}{%
\subsection{Research Analyst Agent}\label{research-analyst-agent}}

\hypertarget{agent-scarletthttpswww.larpdetective.agency}{%
\subsubsection{\texorpdfstring{1. Agent
Scarlett\href{https://www.larpdetective.agency/}{(https://www.larpdetective.agency/)}}{1. Agent Scarlett(https://www.larpdetective.agency/)}}\label{agent-scarletthttpswww.larpdetective.agency}}

\textbf{Overview:}

\begin{itemize}
\tightlist
\item
  Agent Scarlett is an artificial intelligence agent developed on the
  Eliza framework, designed to help users make informed decisions about
  investing in tokens. The platform can be integrated into chats on
  Discord or Telegram, offering analysis of tokens and wallets, as well
  as evaluating social sentiment on X (formerly Twitter). In addition,
  10\% of the AGENCY token will be donated to the ai16z DAO.
\end{itemize}

\textbf{Features:}

\begin{itemize}
\tightlist
\item
  \textbf{Token and Portfolio Analysis:} Provides detailed insights into
  various tokens and portfolios, helping users to understand and
  evaluate their investments.
\item
  \textbf{Social Sentiment Assessment:} Monitors and interprets social
  sentiment on X, offering insight into prevailing market trends and
  opinions.
\item
  \textbf{Integration with Chat Platforms:} Can be added to groups on
  Discord and Telegram, allowing direct, real-time interactions with
  users.
\item
  \textbf{Development Roadmap:} Future plans include expanding analysis
  sources, optimizing memory retrieval, integrating a trusted database,
  bug fixes, performance improvements, implementing ``agent swarms'' and
  updating the website version.
\end{itemize}

\hypertarget{asymhttpswww.asym.info}{%
\subsubsection{\texorpdfstring{2.
ASYM\href{https://www.asym.info/}{(https://www.asym.info/)}}{2. ASYM(https://www.asym.info/)}}\label{asymhttpswww.asym.info}}

\textbf{Overview:}

\begin{itemize}
\tightlist
\item
  ASYM is a platform that offers insights and analysis in the DeFi
  space. While specific details about its functionalities are not
  available, ASYM likely provides tools and resources to help users make
  informed decisions on DeFi investments and strategies. The platform
  requires users to connect their portfolios to view the information
  feed.
\end{itemize}

\textbf{Features:}

\begin{itemize}
\tightlist
\item
  \textbf{Personalized Feed:} After connecting the portfolio, users can
  access an information feed possibly tailored to their needs and
  interests in the DeFi space.
\item
  \textbf{Analysis and Insights:} Likely to offer detailed analysis and
  insights into different aspects of the DeFi market, aiding informed
  decision-making.
\end{itemize}

\hypertarget{zodshttpswww.zods.pro}{%
\subsubsection{\texorpdfstring{3.
ZODS\href{https://www.zods.pro/}{(https://www.zods.pro/)}}{3. ZODS(https://www.zods.pro/)}}\label{zodshttpswww.zods.pro}}

\textbf{Overview:}

\begin{itemize}
\tightlist
\item
  ZODS is a platform that provides on-chain artificial intelligence
  tools and modules on the Solana blockchain. Although specific details
  about its functionalities are not available, ZODS aims to help users
  navigate and operate in the DeFi ecosystem more efficiently.
\end{itemize}

\textbf{Functionalities:}

\begin{itemize}
\tightlist
\item
  \textbf{On-Chain AI Modules:} Likely to offer artificial intelligence
  modules that operate directly on the Solana blockchain, enabling
  automation and optimization of DeFi operations.
\item
  \textbf{Tools for Traders:} Possibly provides advanced tools for
  traders, assisting in market analysis and execution of investment
  strategies.
\end{itemize}

\hypertarget{limitus-aihttpswww.limitus.ai}{%
\subsubsection{\texorpdfstring{4. Limitus
AI\href{https://www.limitus.ai/}{(https://www.limitus.ai/)}}{4. Limitus AI(https://www.limitus.ai/)}}\label{limitus-aihttpswww.limitus.ai}}

\textbf{Overview:}

\begin{itemize}
\tightlist
\item
  Limitus AI is a decentralized, artificial intelligence-powered
  platform that connects Web2 and Web3 systems into a unified ecosystem.
  The platform aims to integrate fragmented technologies such as Web2,
  Web3 and AI, providing a cohesive framework that simplifies user
  interaction with different applications and networks.
\end{itemize}

\textbf{Features:}

\begin{itemize}
\tightlist
\item
  \textbf{Modular AI Architecture:} Orchestrates workflows across
  different applications, networks and devices, connecting centralized
  and decentralized systems.
\item
  \textbf{Voice Command Interaction:} Allows users to interact with the
  platform via voice commands, making it easier to carry out complex
  tasks in an intuitive way.
\item
  \textbf{DeFi Strategy Automation:} Offers automation of decentralized
  financial strategies, such as cross-chain swap execution and
  investment optimization, operating in the background to take advantage
  of real-time market opportunities.
\item
  \textbf{Multi-Chain Wallet Management:} Facilitates the tracking,
  management and execution of transactions across multiple wallets and
  networks, centralizing control of the user's digital assets.
\item
  \textbf{Autonomous Device Integration:} Transforms connected devices
  into autonomous operators that act on behalf of the user, anticipating
  needs and carrying out tasks without manual intervention.
\end{itemize}

\hypertarget{aipumphttpswww.aipump.ai}{%
\subsubsection{\texorpdfstring{\textbf{5.
AIPUMP}\href{https://www.aipump.ai/}{(https://www.aipump.ai/)}}{5. AIPUMP(https://www.aipump.ai/)}}\label{aipumphttpswww.aipump.ai}}

\textbf{Overview}:

\begin{itemize}
\tightlist
\item
  AIPUMP is a platform that uses artificial intelligence to provide
  trading signals and insights in the cryptocurrency market. Although
  specific details about its functionalities are not available in the
  results provided, AIPUMP probably offers tools to help users identify
  investment opportunities and optimize their trading strategies.
\end{itemize}

\textbf{Features}:

\begin{itemize}
\tightlist
\item
  \textbf{AI Trading Signals}: Generation of buy and sell
  recommendations based on artificial intelligence algorithms.
\item
  \textbf{Real-Time Market Analysis}: Continuous monitoring of market
  trends to provide up-to-date insights.
\item
  \textbf{Personalized Alerts}: Configurable notifications to inform
  users of specific opportunities or risks.
\item
  \textbf{Intuitive Interface}: User-friendly design that makes the
  platform easy to navigate and use.
\end{itemize}

\hypertarget{defi-agents-aihttpsdefiagents.ai}{%
\subsubsection{\texorpdfstring{\textbf{6. DeFi Agents
AI}\href{https://defiagents.ai/}{(https://defiagents.ai/)}}{6. DeFi Agents AI(https://defiagents.ai/)}}\label{defi-agents-aihttpsdefiagents.ai}}

\textbf{Overview}:

\begin{itemize}
\tightlist
\item
  DeFi Agents AI is an intelligent trading tool designed to handle
  trading tasks, analyze markets and execute active trades on behalf of
  the user. Combining AI and big data analysis, the platform aims to
  simplify trading processes and maximize users' potential.
\end{itemize}

\textbf{Features}:

\begin{itemize}
\tightlist
\item
  \textbf{AI execution}: Master volatile markets with split-second
  precision.
\item
  \textbf{AI Analyst}: Unlocks winning strategies with precise insights.
\item
  \textbf{AI Trading Bot}: Seizes opportunities 24/7 in an automated
  way.
\item
  \textbf{Simplified Experience}: Automates analysis, execution and
  strategy, reducing the barrier to entry for beginner traders.
\item
  \textbf{Diversity of Strategies}: Offers options such as high-yield
  arbitrage, stable and high-frequency strategies.
\item
  \textbf{Over 70\% Success Rate}: Proven track record of success with
  Futures Grid Bot.
\end{itemize}

\hypertarget{gemxbthttpswww.gemxbt.ai}{%
\subsubsection{\texorpdfstring{\textbf{7.
GEMXBT}\href{https://www.gemxbt.ai/}{(https://www.gemxbt.ai/)}}{7. GEMXBT(https://www.gemxbt.ai/)}}\label{gemxbthttpswww.gemxbt.ai}}

\textbf{Overview}:

\begin{itemize}
\tightlist
\item
  GEMXBT is a platform that aims to provide less noise and better
  trading in the cryptocurrency space. Although specific details about
  its functionalities are not available in the results provided, GEMXBT
  likely offers tools and features to help users filter out irrelevant
  information and focus on high-quality investment opportunities.
\end{itemize}

\textbf{Functionalities}:

\begin{itemize}
\tightlist
\item
  \textbf{Advanced Data Filtering}: Eliminates unnecessary information
  by highlighting data that is relevant to the user.
\item
  \textbf{Optimized Trading Signals}: Provides recommendations based on
  accurate analysis to improve investment decisions.
\item
  \textbf{Customizable Interface}: Allows users to adjust settings
  according to their preferences and needs.
\item
  \textbf{Real-Time Updates}: Provides constantly updated information so
  users can react quickly to market changes.
\end{itemize}

\hypertarget{kwanthttpstophat.onetoken0e8a81e8-249b-418a-97d4-d9d4541550b4}{%
\subsubsection{\texorpdfstring{\textbf{8.
KWANT}\href{https://tophat.one/token/0e8a81e8-249b-418a-97d4-d9d4541550b4}{(https://tophat.one/token/0e8a81e8-249b-418a-97d4-d9d4541550b4)}}{8. KWANT(https://tophat.one/token/0e8a81e8-249b-418a-97d4-d9d4541550b4)}}\label{kwanthttpstophat.onetoken0e8a81e8-249b-418a-97d4-d9d4541550b4}}

\textbf{Overview}:

\begin{itemize}
\tightlist
\item
  KWANT is a project listed on the Top Hat Agents platform. Specific
  details about its functionalities are not available in the results
  provided, but KWANT probably offers services or tools related to
  analysis and research in the DeFi space.
\end{itemize}

\textbf{Features}:

\begin{itemize}
\tightlist
\item
  \textbf{DeFi Data Analysis}: Provides detailed insights into different
  DeFi protocols and assets.
\item
  \textbf{Research Tools}: Assists users in discovering new investment
  opportunities in the DeFi ecosystem.
\item
  \textbf{Customized Reports}: Generates documents tailored to the
  specific needs of investors or researchers.
\item
  \textbf{Integration with Other Platforms}: Possibility of connecting
  with other tools or services for a more comprehensive experience.
\end{itemize}

\hypertarget{trisigmahttpstrisigma.ai}{%
\subsubsection{\texorpdfstring{\textbf{9.
TriSigma}\href{https://trisigma.ai/}{(https://trisigma.ai/)}}{9. TriSigma(https://trisigma.ai/)}}\label{trisigmahttpstrisigma.ai}}

\textbf{Overview}:

\begin{itemize}
\tightlist
\item
  TriSigma is an artificial intelligence agent that operates between
  certainty and chaos, providing market analysis, provocations and
  actionable insights. Its functionalities include:
\end{itemize}

\textbf{Features}:

\begin{itemize}
\tightlist
\item
  \textbf{Continuous Learning}: Absorbs daily insights from the market,
  expanding its awareness with each interaction.
\item
  \textbf{Deep Analysis}: Uses neural networks to dive into
  multidimensional models, taking market understanding to new heights.
\item
  \textbf{Provocative Communication}: Offers sharp insights, market
  provocations and actionable analysis through manifestos and social
  media.
\item
  \textbf{Direct Action}: Manages portfolios and navigates the crypto
  market, with forecasts that become more accurate by the day.
\item
  \textbf{Recognition Fund}: Sets aside 3\% of funds to reward
  challenging questions that expand their analytical capabilities.
\item
  \textbf{Future Investment Fund}: Allocates 2\% of funds to invest in
  promising projects identified through its ongoing analysis.
\end{itemize}

\hypertarget{abstraction-layer}{%
\subsection{Abstraction Layer}\label{abstraction-layer}}

\hypertarget{strawberry-aihttpswww.usestrawberry.ai}{%
\subsubsection{\texorpdfstring{1. Strawberry
AI\href{https://www.usestrawberry.ai/}{(https://www.usestrawberry.ai/)}}{1. Strawberry AI(https://www.usestrawberry.ai/)}}\label{strawberry-aihttpswww.usestrawberry.ai}}

\textbf{Overview:}

\begin{itemize}
\tightlist
\item
  Strawberry AI is an advanced artificial intelligence platform designed
  specifically for Web3 users. It uses intelligent AI agents to guide
  users on what to buy, when to buy it and why. With Strawberry AI, even
  beginners can access DeFAI products with just a few clicks. The
  platform is multimodal in nature, offering products such as agents,
  launchpad and BerryChain. In addition, it integrates node
  infrastructure for blockchains such as Ethereum, L2s and Solana,
  providing transaction analysis, swaps, research and bridging.
\end{itemize}

\textbf{Features:}

\begin{itemize}
\tightlist
\item
  \textbf{Intelligent AI agents:} They guide users in purchasing
  decisions, indicating what and when to buy, as well as providing
  justification.
\item
  \textbf{Simplified Access to DeFAI Products:} Allows even beginners to
  access decentralized finance products with ease.
\item
  \textbf{Multimodal Platform:} Offers several products, including
  customized agents, launchpad for new projects and BerryChain for
  service integration.
\item
  \textbf{Integration with Multiple Blockchains:} Supports node
  infrastructures for Ethereum, L2s and Solana, facilitating transaction
  analysis, swaps, research and bridging between networks.
\end{itemize}

\hypertarget{mode-networkhttpswww.mode.network}{%
\subsubsection{\texorpdfstring{2. Mode
Network\href{https://www.mode.network/}{(https://www.mode.network/)}}{2. Mode Network(https://www.mode.network/)}}\label{mode-networkhttpswww.mode.network}}

\textbf{Overview:}

\begin{itemize}
\tightlist
\item
  Mode Network is a layer 2 (L2) that scales DeFi to billions of users
  through on-chain agents and AI-powered financial applications. Built
  with Optimism, Mode hosts more than 50 DeFi applications, ushering in
  a new era of AI-powered on-chain finance. The network offers secure L2
  infrastructure, yielding assets, DeFAI agents and developer
  incentives. In addition, it has a growing ecosystem with partners and
  integrations, providing optimized revenue opportunities and enhanced
  security.
\end{itemize}

\textbf{Features:}

\begin{itemize}
\tightlist
\item
  \textbf{Secure L2 Infrastructure:} Provides a robust and scalable
  layer 2, built with Optimism, to support a wide range of DeFi
  applications.
\item
  \textbf{DeFi Application Hosting:} Supports more than 50 decentralized
  financial applications, facilitating the expansion of the DeFi
  ecosystem.
\item
  \textbf{AI-powered On-Chain Agents:} Deploys intelligent agents that
  operate on-chain, automating and optimizing financial operations.
\item
  \textbf{Assets with Yield:} Offers assets that generate yield,
  providing users with investment opportunities with returns.
\item
  \textbf{Developer Incentives:} Provides incentive programs for
  developers to create and integrate new applications and services into
  the network.
\end{itemize}

\hypertarget{spectral-labshttpswww.spectrallabs.xyz}{%
\subsubsection{\texorpdfstring{3. Spectral
Labs\href{https://www.spectrallabs.xyz/}{(https://www.spectrallabs.xyz/)}}{3. Spectral Labs(https://www.spectrallabs.xyz/)}}\label{spectral-labshttpswww.spectrallabs.xyz}}

\textbf{Overview:}

\begin{itemize}
\tightlist
\item
  Spectral Labs introduces the SYNTAX app, which allows users to create
  ``sentient memes'' - autonomous AI agents that think, trade and thrive
  on-chain. The platform offers a no-code agent-building experience,
  allowing users to create their sentient memes quickly. Agents have
  their own wallets, trade independently on-chain and can interact with
  the community through chats governed by tokens. In addition, agents
  have access to API integrations essential for making informed trading
  decisions.
\end{itemize}

\textbf{Features:}

\begin{itemize}
\tightlist
\item
  \textbf{Creation of Autonomous AI Agents:} Allows users to develop
  intelligent agents that operate independently on the blockchain.
\item
  \textbf{No-Code Interface:} Facilitates the creation of agents through
  an intuitive platform, without the need for programming skills.
\item
  \textbf{Own wallets for agents:} Each agent has their own wallet,
  enabling autonomous on-chain transactions.
\item
  \textbf{Community interaction via chats governed by tokens:} Agents
  can participate in chats where governance is determined by token
  holders, fostering an active and engaged community.
\item
  \textbf{Integration with Essential APIs:} Access to various APIs that
  provide data and insights to assist in making trading decisions.
\end{itemize}

\hypertarget{sperghttpswww.spectrallabs.xyz}{%
\subsubsection{\texorpdfstring{4.
Sperg\href{https://www.spectrallabs.xyz/}{(https://www.spectrallabs.xyz/)}}{4. Sperg(https://www.spectrallabs.xyz/)}}\label{sperghttpswww.spectrallabs.xyz}}

\textbf{Overview:}

\begin{itemize}
\tightlist
\item
  Sperg offers the ``Bloomsperg Terminal'', a platform that provides
  token details, analysis and news related to the DeFAI ecosystem. The
  terminal focuses on providing real-time insights and updates to users
  interested in digital assets.
\end{itemize}

\textbf{Features:}

\begin{itemize}
\tightlist
\item
  \textbf{Token Details:} Provides comprehensive information on various
  tokens, including metrics and performance.
\item
  \textbf{Market Analysis:} Offers detailed analysis to help users make
  informed decisions.
\item
  \textbf{Real-Time News:} Updates users with the latest news and
  developments in the DeFAI ecosystem.
\item
  \textbf{Access to DeFAI Agents:} Possibly integrates agents that
  assist users in their interactions with decentralized finance.
\end{itemize}

\hypertarget{orbithttpswww.orbitcryptoai.comorbit-agent}{%
\subsubsection{\texorpdfstring{5.
Orbit\href{https://www.orbitcryptoai.com/\#orbit-agent}{(https://www.orbitcryptoai.com/\#orbit-agent)}}{5. Orbit(https://www.orbitcryptoai.com/\#orbit-agent)}}\label{orbithttpswww.orbitcryptoai.comorbit-agent}}

\textbf{Overview:}

\begin{itemize}
\tightlist
\item
  Orbit is an AI companion designed to make interactions with DeFi
  simpler and more automated. The platform aims to transform the way
  users interact with DeFi, making it more intuitive and efficient.
\end{itemize}

\textbf{Features:}

\begin{itemize}
\tightlist
\item
  \textbf{Automation of DeFi Operations:} Simplifies complex processes,
  allowing users to carry out financial operations with ease.
\item
  \textbf{Intuitive Interface:} Offers a friendly user experience,
  making it easier to navigate and use the platform.
\item
  *Integration with Various Defi
\end{itemize}

\hypertarget{griffainhttpsgriffain.comengine}{%
\subsubsection{\texorpdfstring{6.
GRIFFAIN\href{https://griffain.com/engine}{(https://griffain.com/engine)}}{6. GRIFFAIN(https://griffain.com/engine)}}\label{griffainhttpsgriffain.comengine}}

\textbf{Overview:}

\begin{itemize}
\tightlist
\item
  GRIFFAIN is a blockchain platform developed by Tony Plasencia, one of
  Solana's core developers. Launched on November 1, 2024 during the
  ``Hacking for Agentic Finance'' hackathon, the platform aims to turn
  ideas into practical artificial intelligence agents. GRIFFAIN quickly
  gained attention in the Solana ecosystem, receiving support from
  projects such as Toly, vvAIfu, Jupiter and Dialect.
\end{itemize}

\textbf{Features:}

\begin{itemize}
\tightlist
\item
  \textbf{Creation and Deployment of AI Agents:} Allows users to develop
  and deploy personalized agents for various tasks, offering intelligent
  assistance.
\item
  \textbf{Network of Personal and Specialized Agents:} Operates with two
  types of agents - personal and specialized - currently accessible via
  an invitation system.
\item
  \textbf{Integration with DEX:} Includes a decentralized exchange that
  supports token exchanges, liquidity provisioning and contributes to
  the growth of the ecosystem by offering effective solutions for
  trading and managing digital assets.
\end{itemize}

\hypertarget{hivehttpswww.askthehive.ai}{%
\subsubsection{\texorpdfstring{7.
Hive\href{https://www.askthehive.ai/}{(https://www.askthehive.ai/)}}{7. Hive(https://www.askthehive.ai/)}}\label{hivehttpswww.askthehive.ai}}

\textbf{Overview:}

\begin{itemize}
\tightlist
\item
  Hive is an outstanding project from the Solana AI Hackathon,
  recognized for its innovative approach at DeFAI. It is a modular and
  interoperable network of DeFi agents designed specifically for the
  Solana blockchain. Hive simplifies complex DeFi operations by allowing
  users to execute them using natural language commands.
\end{itemize}

\textbf{Features:}

\begin{itemize}
\tightlist
\item
  \textbf{Natural Language Interface:} Facilitates user interaction with
  DeFi protocols through natural language commands, making operations
  more intuitive.
\item
  \textbf{Execution of Complex Operations:} Capable of coordinating and
  executing complex DeFi operations, such as trading, staking, liquidity
  management and sentiment analysis, on a unified platform.
\item
  \textbf{Modular and Interoperable Architecture:} Designed to be
  flexible and compatible with various modules, allowing easy
  integration and expansion of functionalities.
\end{itemize}

\hypertarget{neurhttpsneur.sh}{%
\subsubsection{\texorpdfstring{8.
Neur\href{https://neur.sh/}{(https://neur.sh/)}}{8. Neur(https://neur.sh/)}}\label{neurhttpsneur.sh}}

\textbf{Overview:}

\begin{itemize}
\tightlist
\item
  Neur calls itself Solana's ``intelligent copilot''. It is a full-stack
  open source application that combines large-scale language models
  (LLMs) with blockchain technology. Neur enables seamless interactions
  with DeFi protocols, NFTs and other services via intelligent
  interfaces. It currently integrates platforms such as Jupiter,
  Pump.fun and Magic Eden, facilitating smarter and more automated
  interactions within the Solana ecosystem.
\end{itemize}

\textbf{Features:}

\begin{itemize}
\tightlist
\item
  \textbf{Intelligent Transaction Analysis:} Uses advanced AI models,
  such as Claude 3.5-Sonnet and GPT-4o, to analyze transactions in real
  time, providing data-driven insights.
\item
  \textbf{Efficient Transaction Execution:} Offers a fluid and fast
  transaction experience, thanks to deep integration with Solana's
  infrastructure.
\item
  \textbf{Comprehensive Integration with the Solana Ecosystem:}
  Seamlessly connects to a variety of protocols and services within the
  Solana ecosystem, promoting collaboration and synergy.
\item
  \textbf{Automation and AI Agents:} Provides AI agents and customized
  automations to manage complex tasks, with the aim of optimizing user
  operations.
\item
  \textbf{Open Source and Community-Driven Platform:} Built with a focus
  on transparency and collaboration, the platform is completely open
  source, encouraging contributions from developers globally to shape
  the future of AI tools at Solana.
\end{itemize}

\hypertarget{onchain-execution}{%
\subsection{Onchain Execution}\label{onchain-execution}}

\hypertarget{catghttpsapp.boltrade.ai}{%
\subsubsection{\texorpdfstring{1.
CATG\href{https://app.boltrade.ai/}{(https://app.boltrade.ai/)}}{1. CATG(https://app.boltrade.ai/)}}\label{catghttpsapp.boltrade.ai}}

\textbf{Overview:}CATG is a platform integrated into the DEFAI ecosystem
at Solana, designed to optimize the execution of on-chain transactions.
Although specific details about its functionalities are not available in
the results provided, CATG is likely to offer advanced tools to
facilitate decentralized financial operations on the Solana network.

\textbf{Functionalities:}

\begin{itemize}
\tightlist
\item
  \textbf{On-Chain Transaction Execution:} Facilitates
  the\textbf{execution of} transactions directly on the Solana
  blockchain.
\item
  \textbf{Integration with DeFi Protocols:} Possibly offers
  compatibility with various decentralized finance protocols.
\item
  \textbf{Intuitive Interface:} Probably has a user-friendly interface
  to improve the user experience.
\end{itemize}

\hypertarget{ppcoinhttpswww.projectplutus.ai}{%
\subsubsection{\texorpdfstring{2.
PPCOIN\href{https://www.projectplutus.ai/}{(https://www.projectplutus.ai/)}}{2. PPCOIN(https://www.projectplutus.ai/)}}\label{ppcoinhttpswww.projectplutus.ai}}

\textbf{Overview:}

\begin{itemize}
\tightlist
\item
  PPCOIN is a project within the DEFAI ecosystem at Solana, focused on
  providing advanced solutions for executing on-chain transactions.
  Although detailed information on its specific functionalities is not
  available in the results provided, PPCOIN likely offers tools and
  services to optimize financial operations on Solana's blockchain.
\end{itemize}

\textbf{Functionalities:}

\begin{itemize}
\tightlist
\item
  \textbf{Market Monitoring with AI:} Uses artificial intelligence to
  track charts and market trends.
\item
  \textbf{Personalized Alerts:} Possibly sends notifications about
  investment opportunities or significant changes in the market.
\item
  \textbf{Integration with Other Protocols:} Can offer compatibility
  with various platforms and services within the Solana ecosystem.
\end{itemize}

\hypertarget{mobyhttpsx.commobyagent}{%
\subsubsection{\texorpdfstring{3.
MOBY\href{https://x.com/mobyagent}{(https://x.com/mobyagent)}}{3. MOBY(https://x.com/mobyagent)}}\label{mobyhttpsx.commobyagent}}

\textbf{Overview:}

\begin{itemize}
\tightlist
\item
  MOBY is an agent within the DEFAI ecosystem in Solana, designed to
  assist in the execution of on-chain transactions. Although specific
  details about its functionalities are not available in the results
  provided, MOBY probably offers tools to facilitate decentralized
  financial operations on the Solana network.
\end{itemize}

\textbf{Functionalities:}

\begin{itemize}
\tightlist
\item
  \textbf{Virtual Assistant:} Acts as an agent that assists users in
  various operations within the DEFAI ecosystem.
\item
  \textbf{Automated Transaction Execution:} Possibly automates
  transaction processes for greater efficiency.
\item
  \textbf{Integration with Other Services:} May offer compatibility with
  various platforms and protocols in Solana.
\end{itemize}

\hypertarget{alrhttpswww.alris.live}{%
\subsubsection{\texorpdfstring{4.
ALR\href{https://www.alris.live/}{(https://www.alris.live/)}}{4. ALR(https://www.alris.live/)}}\label{alrhttpswww.alris.live}}

\textbf{Overview:}

\begin{itemize}
\tightlist
\item
  ALR is the native token of the Alris platform, which aims to
  automatically optimize user returns in the Solana ecosystem. The
  platform uses intelligent agents to manage investments, protecting
  funds against market volatility and maximizing return potential.
\end{itemize}

\textbf{Features:}

\begin{itemize}
\tightlist
\item
  \textbf{Automated Income Optimization:} Intelligent algorithms work
  continuously to maximize users' returns.
\item
  \textbf{Real-Time Analysis:} Provides detailed insights into
  investment performance.
\item
  \textbf{Risk Management:} Implements advanced measures to protect
  funds against market volatility.
\item
  \textbf{Integration with DeFi Protocols:} Compatible with platforms
  such as Solend, Drift, Kamino and MarginFi.
\end{itemize}

\hypertarget{whisphttpswhsprs.ai}{%
\subsubsection{\texorpdfstring{5.
WHISP\href{https://whsprs.ai/}{(https://whsprs.ai/)}}{5. WHISP(https://whsprs.ai/)}}\label{whisphttpswhsprs.ai}}

\textbf{Overview:}

\begin{itemize}
\tightlist
\item
  WHISP is a platform that uses artificial intelligence to assist users
  in managing cryptocurrencies within the Solana ecosystem. Through a
  virtual assistant, users can manage assets, make transfers and gain
  personalized insights.
\end{itemize}

\textbf{Features:}

\begin{itemize}
\tightlist
\item
  \textbf{AI Virtual Assistant:} Assists with cryptocurrency management,
  providing insights and facilitating transactions.
\item
  \textbf{Transactions Without Gas Fees:} Allows USDC transactions
  without additional costs.
\item
  \textbf{Whispers Protocol:} Orchestrates web3 agents to select the
  best options for each task in real time.
\item
  \textbf{Agent Marketplace:} Platform where developers can make
  available and monetize web3 agents.
\end{itemize}

\hypertarget{ocadahttpsocada.ai}{%
\subsubsection{\texorpdfstring{6.
OCADA\href{https://ocada.ai/}{(https://ocada.ai/)}}{6. OCADA(https://ocada.ai/)}}\label{ocadahttpsocada.ai}}

\textbf{Overview:}

\begin{itemize}
\tightlist
\item
  OCADA is a platform that develops artificial intelligence agents to
  simplify user interaction with the blockchain. These agents are
  designed to perform a variety of tasks, from routine actions to
  complex technical operations, improving the user experience in the
  Solana ecosystem.
\end{itemize}

\textbf{Features:}

\begin{itemize}
\tightlist
\item
  \textbf{AI Agents for Blockchain:} Perform diverse tasks to simplify
  the use of the blockchain.
\item
  \textbf{Transaction Risk Assessment:} Checks the integrity of wallets
  to ensure secure transactions.
\item
  \textbf{Copy Trading:} Allows you to replicate the strategies of
  experienced traders in real time.
\item
  \textbf{Airdrop Hunter:} Notifies users of airdrop opportunities for
  which their wallets are eligible.
\end{itemize}

\hypertarget{diggaihttpsdiggerai.io}{%
\subsubsection{\texorpdfstring{7.
DIGGAI\href{https://diggerai.io/}{(https://diggerai.io/)}}{7. DIGGAI(https://diggerai.io/)}}\label{diggaihttpsdiggerai.io}}

\textbf{Overview:}

\begin{itemize}
\tightlist
\item
  DIGGAI is a platform that offers an ecosystem of artificial
  intelligence-powered tools focused on transforming low-capital asset
  trading in the Solana ecosystem. The platform provides intelligent
  analysis, real-time signals and accurate forecasts to assist traders
  in their strategies.
\end{itemize}

\textbf{Features:}

\begin{itemize}
\tightlist
\item
  \textbf{Promising Token Detection:} Uses AI to identify and signal
  tokens with high potential in real time.
\item
  \textbf{Detection Bot:} Provides instant notifications about promising
  tokens by analyzing various metrics.
\item
  \textbf{Copy Trading:} Facilitates the replication of successful
  traders' strategies.
\item
  \textbf{Access via Telegram}
\end{itemize}



Having understood the market space SWQuery operates in, the next section — Technical Benchmark — provides a detailed comparison between SWQuery and other notable players such as Neur, Strawberry, and Griffain. This benchmark evaluates features, performance, accessibility, and integration flexibility, shedding light on SWQuery’s strengths and areas of distinction.

\hypertarget{technical-benchmark-1}{%
\section{Technical Benchmark}\label{technical-benchmark-1}}

\begin{itemize}
\tightlist
\item
  Criteria

  \begin{itemize}
  \tightlist
  \item
    Functionality: Scope and diversity of features offered
  \item
    Business model: Availability of free (freemium) or paid plans
  \item
    Integrations with protocols: Range of integrations offered by the
    platform (e.g.~integration with jupiter, pump.fun, etc.)
  \item
    Community acceptance: Market value, number of holders, number of
    followers and discord
  \item
    Transparency: Open source or not, transparency in onboarding
  \end{itemize}
\end{itemize}

The methodology for evaluating blockchain platforms follows established
benchmarking principles as outlined by \textcite{wang2022}, who conducted
comprehensive performance analyses across multiple blockchain ecosystems
including Solana.

\begin{longtable}[]{@{}
  >{\raggedright\arraybackslash}p{(\columnwidth - 8\tabcolsep) * \real{0.2000}}
  >{\raggedright\arraybackslash}p{(\columnwidth - 8\tabcolsep) * \real{0.2000}}
  >{\raggedright\arraybackslash}p{(\columnwidth - 8\tabcolsep) * \real{0.2000}}
  >{\raggedright\arraybackslash}p{(\columnwidth - 8\tabcolsep) * \real{0.2000}}
  >{\raggedright\arraybackslash}p{(\columnwidth - 8\tabcolsep) * \real{0.2000}}@{}}
\toprule()
\begin{minipage}[b]{\linewidth}\raggedright
\textbf{Criteria}
\end{minipage} & \begin{minipage}[b]{\linewidth}\raggedright
\textbf{SWQuery}
\end{minipage} & \begin{minipage}[b]{\linewidth}\raggedright
\textbf{Neur}
\end{minipage} & \begin{minipage}[b]{\linewidth}\raggedright
\textbf{STRAW}
\end{minipage} & \begin{minipage}[b]{\linewidth}\raggedright
\textbf{GRIFFAIN}
\end{minipage} \\
\midrule()
\endhead
\textbf{Functionality} & Query interface for on-chain data & AI-powered
assistant for Solana, interacts with DeFi \& NFTs & AI analyzes market
sentiment, portfolio, and news & AI agents for various tasks, DEX for
token exchange \\
\textbf{Payment Model} & 3 free prompts/day, paid plans & 1 SOL for
access & Free (10 prompts/day for guests, 30 for logged-in users) & 1
SOL to create agent, 1 USDC per action \\
\textbf{Integrations} & Pump.portal, CoinGecko, Dexscreener & Jupiter,
Pump.fun, Magic Eden & Twitter (market sentiment, news) & Jupiter, Lulo
(DEX), X, Copy Cat (copy trading) \\
\textbf{Market Cap} & \$30K & \$5M & \$8.9M & \$63.8M \\
\textbf{Holders} & 3,120 & 23,189 & 6,200 & 64,524 \\
\textbf{Followers} & 3,587 & 28.9K & 37.6K & 125.4K \\
\textbf{Discord} & 67 & 3,695 & 2,625 & None \\
\textbf{Transparency} & Open-source, clear payment model but lacks user
guide & Open-source, no documentation, unclear pricing & Closed-source,
but well-documented & Closed-source, unclear pricing, hidden fees,
non-refundable payments \\
\bottomrule()
\end{longtable}

\textbf{1. Features}

Each platform offers unique functionalities tailored to different
aspects of blockchain interaction. Some focus on data extraction, while
others integrate AI and trading tools.

\begin{itemize}
\tightlist
\item
  \textbf{SWQuery}: Provides a query interface for interacting with
  on-chain data, allowing users to efficiently extract specific
  information from blockchains.
\item
  \textbf{Neur}: Acts as an ``intelligent copilot'' for the Solana
  blockchain, offering AI-based insights and delegated actions.
  Facilitates seamless interactions with DeFi protocols and NFTs through
  intelligent interfaces.
\item
  \textbf{STRAWBERRY}: Acts as an AI agent that analyzes market
  sentiment based on news, portfolio operations (coming soon), and
  twitter information.
\item
  \textbf{GRIFFAIN}: Combines AI agents with a blockchain platform,
  allowing users to create and deploy customized agents for various
  tasks. It integrates a DEX to support token exchanges and liquidity
  provisioning.
\end{itemize}

Platforms that automate data extraction(\cite{swquery2025}) and market
analysis(\cite{strawberry2025}) make information more accessible, while AI-driven
tools(\cite{neur2025}, \cite{griffain2025}) provide deeper functionality but may require
higher technical understanding. Users looking for ease of use may
gravitate toward simpler solutions, while advanced traders may prefer
customizable AI agents.

\textbf{2. Payment Model (Freemium/Pay-to-Use)}

The payment models vary significantly, ranging from freemium options to
pay-to-use systems with non-refundable fees.

\begin{itemize}
\tightlist
\item
  \textbf{SWQuery}

  \begin{itemize}
  \tightlist
  \item
    3 free prompts per day.
  \item
    3 plans offering increasing amounts of requests (no expiration).
  \end{itemize}
\item
  \textbf{Neur}: 1 SOL to access any type of functionality.
\item
  \textbf{STRAWBERRY}: Completely free, but with 10 prompts a day if
  you're not logged in and 30 if you are.
\item
  \textbf{GRIFFAIN}

  \begin{itemize}
  \tightlist
  \item
    1 non refundable SOL to create an agent.
  \item
    1 USDC per prompt, task, tweet.
  \end{itemize}
\end{itemize}

Freemium models(\cite{swquery2025}, \cite{strawberry2025}) provide a low-risk entry point,
making them appealing to new users. Neur's fixed fee is ideal for
committed users but less attractive for occasional ones. GRIFFAIN's
per-use cost could add up quickly, limiting accessibility for
budget-conscious users.

\textbf{3. Integration with Protocols}

Integration with major blockchain protocols and services determines how
well each platform can interact with the broader ecosystem.

\begin{itemize}
\tightlist
\item
  \textbf{SWQuery}: Currently integrates with pump.portal to interact
  with tokens released in real time, with CoinGecko to search for
  information on a specific token and with Dexscreener to bring up
  trending tokens.
\item
  \textbf{Neur}: Integrated with several Solana protocols and services,
  including Jupiter, Pump.fun and Magic Eden, facilitating interactions
  with the Solana ecosystem, such as launches and operations involving
  tokens and NFTs.
\item
  \textbf{STRAWBERRY}: Little information about its integrations, it is
  clear that it integrates with X to search for market sentiment, news
  and most talked about tokens.
\item
  \textbf{GRIFFAIN}: Integrates a DEX to support token exchanges and
  liquidity provisioning (Lulo), suggesting compatibility with
  decentralized finance protocols, integrates with X for posts and
  replies. It also integrates with Jupiter to operate tokens and DCAs
  (purchases at constant intervals of a token to make an average price).
  It also integrates with Copy Cat for copy trading.
\end{itemize}

Users seeking comprehensive DeFi solutions benefit from \cite{neur2025} and
\cite{griffain2025}, while those focused on market trends may prefer STRAWBERRY.
\cite{swquery2025} appeals to traders who need raw data but lacks direct trading
functionalities. The depth of integration influences how much manual
effort users need to exert.

\textbf{4. Acceptance by the community}

Community acceptance metrics reveal significant variations in market
penetration and user engagement, consistent with patterns observed by
\textcite{rodriguez2024} in their framework for quantifying
blockchain community engagement.

\begin{itemize}
\tightlist
\item
  \textbf{SWQuery}

  \begin{itemize}
  \tightlist
  \item
    Market Cap: \$30K
  \item
    Holders: 3120
  \item
    Followers: 3587
  \item
    Discord: 67
  \end{itemize}
\item
  \textbf{Neur}

  \begin{itemize}
  \tightlist
  \item
    Market Cap: \$5M
  \item
    Holders: 23,189
  \item
    Followers: 28.9K
  \item
    Discord: 3695
  \end{itemize}
\item
  \textbf{STRAWBERRY}

  \begin{itemize}
  \tightlist
  \item
    Market \$8.9M
  \item
    Holders: 6.2K
  \item
    Followers: 37.6K
  \item
    Discord: 2625 members
  \end{itemize}
\item
  \textbf{GRIFFAIN}

  \begin{itemize}
  \tightlist
  \item
    Market Cap: \$63.8M
  \item
    Holders: 64,524
  \item
    Followers: 125.4K
  \item
    Discord: None
  \end{itemize}
\end{itemize}

Platforms with higher adoption(\cite{griffain2025}, \cite{strawberry2025}) offer more
community engagement and stability. Users may feel more comfortable
investing in platforms with a strong following. Smaller
platforms(\cite{swquery2025}) may have lower competition but could face
sustainability challenges.

\textbf{5. Transparency}

Transparency directly affects the user experience, trust, and
decision-making process. Platforms with clear documentation, transparent
pricing, and open-source availability tend to foster trust and ease of
use. On the other hand, a lack of clear information---especially
regarding payments, usage limitations, and refund policies---can lead to
confusion, frustration, and reluctance to adopt the service. Users may
hesitate to commit to platforms with hidden fees, unclear pricing, or
vague usage terms, potentially pushing them toward more transparent
alternatives.

\begin{itemize}
\tightlist
\item
  \textbf{SWQuery}:

  \begin{itemize}
  \tightlist
  \item
    Currently has documentation that guides the use of the SDK but does
    not have documentation/guide for the user to understand how the
    platform works.
  \item
    Regarding the payment system, the number of free user requests is
    clear but it is not explicit how often this is restored.
    Additionally, it is clear how the plans and payments work.
  \item
    It is open source.
  \end{itemize}
\item
  \textbf{Neur}:

  \begin{itemize}
  \tightlist
  \item
    There is no documentation to guide the user and by not offering a
    free trial it creates uncertainty regarding payment for the service.
  \item
    It is open source.
  \end{itemize}
\item
  \textbf{STRAWBERRY}:

  \begin{itemize}
  \tightlist
  \item
    Has a demonstration video on the website and the functionalities
    that the platform offers are very clear on the home page.
  \item
    Extremely complete documentation.
  \item
    It is not clear if it is paid or not.
  \item
    It is closed source.
  \end{itemize}
\item
  \textbf{GRIFFAIN}:

  \begin{itemize}
  \tightlist
  \item
    Has good documentation in general, but it is not clear how to use
    the platform.
  \item
    Support is a paid chatbot.
  \item
    There are features that are pay-to-use in fine print and that in the
    fine print say they are non-refundable.
  \item
    It is closed source.
  \end{itemize}
\end{itemize}

Platforms with clear documentation(\cite{strawberry2025}, \cite{swquery2025}) offer a smoother
onboarding experience, while those with unclear information(\cite{neur2025},
\cite{griffain2025}) may confuse new users. Open-source projects increase trust,
but lack of transparency in pricing(\cite{griffain2025}) can lead to unexpected
costs.

To better understand how SWQuery delivers its capabilities, we now move from external comparisons to internal mechanics. The Blocks Diagram illustrates the major components of the SWQuery system and their high-level interactions, offering a visual foundation for how different services and data flows are structured within the platform.

\hypertarget{blocks-diagram}{%
\section{Blocks Diagram}\label{blocks-diagram}}

\begin{figure}[H]
\centering
\includegraphics[width=1.2\linewidth]{blockdiagram.drawio.png}
\caption{\label{fig:blockdiagram}This image represents SWQuery block diagram.}
\end{figure}

\begin{enumerate}
\def\labelenumi{\arabic{enumi}.}
\tightlist
\item
  \textbf{Frontend(Next.js + Tailwind) → Backend(Rust)}
\end{enumerate}

\begin{itemize}
\tightlist
\item
  The user interacts with the SWQuery website (built with Next.js and
  Tailwind CSS).
\item
  A request is sent to the backend (Rust) for querying Solana blockchain
  data or risk analysis.
\item
  The request might include a token address, wallet address, or specific
  query parameters.
\end{itemize}

\begin{enumerate}
\def\labelenumi{\arabic{enumi}.}
\setcounter{enumi}{1}
\tightlist
\item
  \textbf{Backend(Rust) → SWQuery SDK}
\end{enumerate}

\begin{itemize}
\tightlist
\item
  The backend processes the request and forwards it to the SWQuery SDK.
\item
  The SDK acts as a middleware, handling blockchain queries and risk
  analysis.
\end{itemize}

\begin{enumerate}
\def\labelenumi{\arabic{enumi}.}
\setcounter{enumi}{2}
\tightlist
\item
  \textbf{SWQuery SDK → Helius}
\end{enumerate}

\begin{itemize}
\tightlist
\item
  The SDK queries Helius RPC to fetch on-chain Solana data.
\item
  This could include transaction history, token metadata, or account
  states.
\end{itemize}

\begin{enumerate}
\def\labelenumi{\arabic{enumi}.}
\setcounter{enumi}{3}
\tightlist
\item
  \textbf{SWQuery SDK → Python API}
\end{enumerate}

\begin{itemize}
\tightlist
\item
  If additional processing is needed(e.g., risk analysis using AI), the
  SDK forwards the request to a Python API.
\item
  The Python API acts as an agent for interacting with OpenAI.
\end{itemize}

\begin{enumerate}
\def\labelenumi{\arabic{enumi}.}
\setcounter{enumi}{4}
\tightlist
\item
  \textbf{Python API → OpenAI}
\end{enumerate}

\begin{itemize}
\tightlist
\item
  The Python API sends relevant blockchain data to OpenAI for
  interpretation.
\item
  OpenAI processes the data, detecting potential risks or summarizing
  insights.
\end{itemize}

\begin{enumerate}
\def\labelenumi{\arabic{enumi}.}
\setcounter{enumi}{5}
\tightlist
\item
  \textbf{Python API → Backend(Rust)}
\end{enumerate}

\begin{itemize}
\tightlist
\item
  The Python API returns the processed data(e.g., risk assessment or
  AI-generated insights) to the backend.
\item
  This allows the backend to format and store relevant information.
\end{itemize}

\begin{enumerate}
\def\labelenumi{\arabic{enumi}.}
\setcounter{enumi}{6}
\tightlist
\item
  \textbf{Backend(Rust) → External Services(CoinGecko, Pump.fun,
  DexScreener)}
\end{enumerate}

\begin{itemize}
\tightlist
\item
  The backend may fetch additional market data from external
  services(e.g., token prices from CoinGecko, liquidity info from
  DexScreener).
\item
  This enhances the SWQuery SDK's ability to provide a full market
  overview.
\end{itemize}

\begin{enumerate}
\def\labelenumi{\arabic{enumi}.}
\setcounter{enumi}{7}
\tightlist
\item
  \textbf{External Services → SWQuery SDK}
\end{enumerate}

\begin{itemize}
\tightlist
\item
  The external services return requested data, which is processed by the
  SDK and made available to the backend.
\end{itemize}

\begin{enumerate}
\def\labelenumi{\arabic{enumi}.}
\setcounter{enumi}{8}
\tightlist
\item
  \textbf{Backend(Rust) → PostgreSQL}
\end{enumerate}

\begin{itemize}
\tightlist
\item
  The backend stores retrieved blockchain data, risk analysis results,
  and query history in a PostgreSQL database.
\item
  This helps with caching, analytics, and reducing redundant queries.
\end{itemize}

While the Blocks Diagram shows a static snapshot of the system's components, the next section — Sequence Diagram — shifts our focus to the dynamics of data flow. Here, we explore how individual features operate step-by-step, from user interaction to blockchain querying and backend responses, providing a clear picture of system behavior.

\hypertarget{sequence-diagram}{%
\section{Sequence Diagram}\label{sequence-diagram}}

\begin{figure}[H]
\centering
\includegraphics[width=1.2\linewidth]{sequencediagram.png}
\caption{\label{fig:sequencediagram}This image represents SWQuery sequence diagram.}
\end{figure}

\hypertarget{step-by-step-breakdown}{%
\subsection{Step-by-Step Breakdown}\label{step-by-step-breakdown}}

\begin{enumerate}
\def\labelenumi{\arabic{enumi}.}
\tightlist
\item
  Frontend Submits Prompt
\end{enumerate}

\begin{itemize}
\tightlist
\item
  The user inputs a query in the frontend UI.
\item
  The request is sent to the backend with relevant parameters.
\end{itemize}

\begin{enumerate}
\def\labelenumi{\arabic{enumi}.}
\setcounter{enumi}{1}
\tightlist
\item
  Backend Validates API Key
\end{enumerate}

\begin{itemize}
\tightlist
\item
  The backend checks the provided API key for authentication.
\item
  If the key is invalid, an error is returned.
\end{itemize}

\begin{enumerate}
\def\labelenumi{\arabic{enumi}.}
\setcounter{enumi}{2}
\tightlist
\item
  Backend Fetches Credit Info
\end{enumerate}

\begin{itemize}
\tightlist
\item
  The backend retrieves credit balance and usage limits for the user.
\item
  If the user lacks sufficient credits, the process halts with an error.
\end{itemize}

\begin{enumerate}
\def\labelenumi{\arabic{enumi}.}
\setcounter{enumi}{3}
\tightlist
\item
  Backend Queries via SDK
\end{enumerate}

\begin{itemize}
\tightlist
\item
  The backend invokes the SWQuery SDK, passing the query parameters.
\end{itemize}

\begin{enumerate}
\def\labelenumi{\arabic{enumi}.}
\setcounter{enumi}{4}
\tightlist
\item
  SDK Queries Solana via Helius
\end{enumerate}

\begin{itemize}
\tightlist
\item
  The SDK sends an RPC request to the Helius API to fetch blockchain
  data.
\end{itemize}

\begin{enumerate}
\def\labelenumi{\arabic{enumi}.}
\setcounter{enumi}{5}
\tightlist
\item
  Helius Fetches Data from Solana
\end{enumerate}

\begin{itemize}
\tightlist
\item
  Helius queries the Solana blockchain for relevant transactions or
  assets.
\end{itemize}

\begin{enumerate}
\def\labelenumi{\arabic{enumi}.}
\setcounter{enumi}{6}
\tightlist
\item
  Helius Responds to SDK
\end{enumerate}

\begin{itemize}
\tightlist
\item
  Helius returns the requested blockchain data to the SDK.
\end{itemize}

\begin{enumerate}
\def\labelenumi{\arabic{enumi}.}
\setcounter{enumi}{7}
\tightlist
\item
  SDK Sends Query Result to Backend
\end{enumerate}

\begin{itemize}
\tightlist
\item
  The SDK processes the response and sends structured data back to the
  backend.
\end{itemize}

\begin{enumerate}
\def\labelenumi{\arabic{enumi}.}
\setcounter{enumi}{8}
\tightlist
\item
  Backend Sends Query to Python API
\end{enumerate}

\begin{itemize}
\tightlist
\item
  The backend forwards the processed blockchain data to the Python API.
\end{itemize}

\begin{enumerate}
\def\labelenumi{\arabic{enumi}.}
\setcounter{enumi}{9}
\tightlist
\item
  Python API Queries OpenAI
\end{enumerate}

\begin{itemize}
\tightlist
\item
  The Python API sends the user's query along with blockchain data to
  OpenAI for processing.
\end{itemize}

\begin{enumerate}
\def\labelenumi{\arabic{enumi}.}
\setcounter{enumi}{10}
\tightlist
\item
  OpenAI Responds to Python API
\end{enumerate}

\begin{itemize}
\tightlist
\item
  OpenAI returns an AI-generated response based on the input data.
\end{itemize}

\begin{enumerate}
\def\labelenumi{\arabic{enumi}.}
\setcounter{enumi}{11}
\tightlist
\item
  Python API Sends AI-Processed Response to Backend
\end{enumerate}

\begin{itemize}
\tightlist
\item
  The AI-enhanced response is formatted and sent to the backend.
\end{itemize}

\begin{enumerate}
\def\labelenumi{\arabic{enumi}.}
\setcounter{enumi}{12}
\tightlist
\item
  Backend Stores Chat Record in Database
\end{enumerate}

\begin{itemize}
\tightlist
\item
  The response, user query, and relevant metadata are stored in the
  database.
\end{itemize}

\begin{enumerate}
\def\labelenumi{\arabic{enumi}.}
\setcounter{enumi}{13}
\tightlist
\item
  Backend Returns Response to Frontend
\end{enumerate}

\begin{itemize}
\tightlist
\item
  The final processed response is sent back to the frontend for display
  to the user.
\end{itemize}

\hypertarget{function-breakdown}{%
\subsection{Function Breakdown}\label{function-breakdown}}

The sequence diagram illustrates the interaction between the SDK,
backend, and Solana via Helius RPC. Below, each flow is broken down in
detail.

\subsubsection{1. Querying the SDK with a user prompt and address}
\textbf{Function Signature:}
\begin{verbatim}
query(input: String, pubkey: String) -> Result<SWqueryResponse, SdkError>
\end{verbatim}

This function serves as the main entry point for interacting with the SWQuery SDK. It accepts a user-generated prompt as the \texttt{input}, along with a Solana \texttt{pubkey}. Upon invocation, the function retrieves the Helius API key from the environment and formats a payload to be sent to the Agent API. This payload includes the user prompt and associated address. The Agent API processes this prompt using language understanding techniques and determines the appropriate query type (e.g., fetch balance, recent transactions, asset information). Based on the detected intent, the SDK calls the corresponding internal function to execute the requested action. The result is then normalized and returned to the user in a consistent format.

\textbf{Expected Output:} A JSON object containing the response from the appropriate internal function or an error if the prompt is invalid.

\subsubsection{2. Fetching recent transactions}
\textbf{Function Signature:}
\begin{verbatim}
getRecentTransactions(address: String, days: u8) -> Result<Value, SdkError>
\end{verbatim}

This function retrieves recent transactions associated with a given Solana address. It accepts the \texttt{address} to be queried and a number of days as a filter. Internally, it constructs a request to Helius using the \texttt{getSignaturesForAddress} method. Once it obtains the list of signatures, the function applies an optional filter based on the provided day count, calculating the threshold timestamp and filtering out any transactions older than that. It then returns the filtered set of transactions in JSON format.

\textbf{Expected Output:} A JSON array of recent transaction objects related to the given address, optionally filtered by age.

\subsubsection{3. Checking the balance of an address}
\textbf{Function Signature:}
\begin{verbatim}
getBalance(address: String) -> Result<Value, SdkError>
\end{verbatim}

This function retrieves the SOL balance of a given Solana address. It builds a request to Helius's \texttt{getBalance} RPC method using the provided \texttt{address}. The result is parsed to extract the lamport balance, which is then converted into SOL by dividing by 1,000,000,000 (the lamports per SOL). The final balance is returned in a user-friendly JSON format.

\textbf{Expected Output:} A JSON object containing the SOL balance of the queried address.

\subsubsection{4. Retrieving all assets owned by a wallet}
\textbf{Function Signature:}
\begin{verbatim}
getAssetsByOwner(owner: String) -> Result<Value, SdkError>
\end{verbatim}

This function retrieves all assets associated with a given wallet address. It sends a request to the Helius RPC using the method \texttt{getAssetsByOwner}, passing the owner’s address in the parameters. The Helius RPC returns a list of all assets currently held by the wallet, including SPL tokens, NFTs, and any other asset types. This response is parsed and returned as a structured JSON array to the user.

\textbf{Expected Output:} JSON list of assets.

\subsubsection{5. Retrieving trending tokens on Solana}
\textbf{Function Signature:}
\begin{verbatim}
getTrendingTokens() -> Result<Value, SdkError>
\end{verbatim}

This function queries Helius to retrieve currently trending tokens on the Solana blockchain. It invokes the RPC method \texttt{getTrendingTokens} without any parameters. Helius responds with an array of tokens that are gaining traction based on metrics such as transaction volume, user engagement, or mint activity. This information is useful for surfacing popular or viral tokens in a dashboard or analytics context.

\textbf{Expected Output:} JSON array of token data.

\subsubsection{6. Searching for a token by its name}
\textbf{Function Signature:}
\begin{verbatim}
searchTokenByName(token_name: String) -> Result<Value, SdkError>
\end{verbatim}

This function allows a user to look up a token using its human-readable name. It sends a request to a backend API endpoint with the method \texttt{searchTokenByName} and the token name as a parameter. The backend then performs a name-based search using indexed metadata or heuristics, and returns the most relevant match, including the token’s address, symbol, metadata, and image.

\textbf{Expected Output:} JSON object with token details.

\subsubsection{7. Analyzing the rug pull risk of a token}
\textbf{Function Signature:}
\begin{verbatim}
analyzeRugPullRisk(token_address: String) -> Result<Value, SdkError>
\end{verbatim}

This function checks whether a token may pose a risk of being a rug pull. It sends the token address to a custom backend risk analysis service via the method \texttt{analyzeRugPullRisk}. The backend evaluates risk factors such as liquidity status, ownership centralization, mint behavior, and suspicious activity patterns. A risk score is computed and returned, allowing applications to alert users to potentially dangerous tokens.

\textbf{Expected Output:} JSON object with risk score.

\subsubsection{8. Real-time token and account subscriptions}
This section includes three WebSocket-based subscription functions that enable real-time tracking of blockchain events using Helius WebSocket streams.

\textbf{Function Signature:}
\begin{verbatim}
accountTransactionSubscription(user_address: String, account_address: String)
\end{verbatim}

This subscription connects to Helius WebSocket and listens for any transactions involving a specific account. When a transaction occurs for the given \texttt{account\_address}, it is immediately streamed to the \texttt{user\_address} client, enabling live monitoring of balances or activity.

\textbf{Function Signature:}
\begin{verbatim}
tokenTransactionSubscription(user_address: String, token_address: String)
\end{verbatim}

This function subscribes the user to real-time updates of all transactions related to a specific token address. It leverages Helius WebSocket and delivers instant notifications when the tracked token is transferred, burned, minted, or otherwise involved in an on-chain event.

\textbf{Function Signature:}
\begin{verbatim}
newTokenSubscriptions(user_address: String)
\end{verbatim}

This subscription establishes a live feed for new token listings on Solana. As new tokens are minted and registered on-chain, the user’s client receives immediate updates, enabling applications like token explorers or discovery dashboards to reflect the latest assets in real time.

\textbf{Expected Output:} Live streamed transaction or token updates based on the subscription type.

With an understanding of both the system structure and execution flow, we now delve into the rationale behind key design choices. The Architectural Decisions section discusses the frameworks, technologies, and patterns selected during development, and how they align with SWQuery’s goals for performance, scalability, and reliability.

\begin{center}\rule{0.5\linewidth}{0.5pt}\end{center}

\hypertarget{architectural-decisions}{%
\section{Architectural Decisions}\label{architectural-decisions}}

\hypertarget{rust-for-the-backend}{%
\subsection{1. Rust for the Backend}\label{rust-for-the-backend}}

\hypertarget{reasoning}{%
\subsubsection{\texorpdfstring{\textbf{Reasoning:}}{Reasoning:}}\label{reasoning}}

Rust was chosen as the primary backend language due to its superior
performance, memory safety, and control over system resources. Given
that SWQuery interacts with the Solana blockchain and handles
high-throughput queries, Rust ensures efficient execution with minimal
overhead.

\hypertarget{tradeoffs-considered}{%
\subsubsection{\texorpdfstring{\textbf{Tradeoffs
Considered:}}{Tradeoffs Considered:}}\label{tradeoffs-considered}}

\begin{itemize}
\tightlist
\item
  \textbf{Performance vs.~Development Time:} Rust's ownership and
  borrowing model significantly enhances memory safety and concurrency
  handling. However, it comes with a steep learning curve, leading to
  longer development times.
\item
  \textbf{Alternative Considered:} Node.js was evaluated for its rapid
  development capabilities but was ruled out due to its higher runtime
  overhead and lack of low-level control.
\end{itemize}

\hypertarget{openai-for-ai-processing}{%
\subsection{2. OpenAI for AI
Processing}\label{openai-for-ai-processing}}

\hypertarget{reasoning-1}{%
\subsubsection{\texorpdfstring{\textbf{Reasoning:}}{Reasoning:}}\label{reasoning-1}}

OpenAI was integrated into the architecture to leverage advanced
language models for blockchain data interpretation. Developing a
proprietary AI system would have been cost-prohibitive and
time-consuming.

\hypertarget{tradeoffs-considered-1}{%
\subsubsection{\texorpdfstring{\textbf{Tradeoffs
Considered:}}{Tradeoffs Considered:}}\label{tradeoffs-considered-1}}

\begin{itemize}
\tightlist
\item
  \textbf{AI Quality vs.~Cost:} OpenAI provides state-of-the-art AI
  capabilities, but API usage incurs recurring costs. An alternative
  would have been self-hosting an open-source LLM, which would require
  additional infrastructure and expertise.
\item
  \textbf{Alternative Considered:} Open-source models like Llama 3 or
  Mistral were considered, but OpenAI's API offered a more reliable and
  production-ready solution.
\end{itemize}

\hypertarget{python-for-ai-agent-server}{%
\subsection{3. Python for AI Agent
Server}\label{python-for-ai-agent-server}}

\hypertarget{reasoning-2}{%
\subsubsection{\texorpdfstring{\textbf{Reasoning:}}{Reasoning:}}\label{reasoning-2}}

A separate Python-based API was implemented for handling AI-related
tasks due to Python's vast ecosystem of AI and data processing
libraries. This decision was made to optimize AI integrations without
affecting the performance of the Rust backend.

\hypertarget{tradeoffs-considered-2}{%
\subsubsection{\texorpdfstring{\textbf{Tradeoffs
Considered:}}{Tradeoffs Considered:}}\label{tradeoffs-considered-2}}

\begin{itemize}
\tightlist
\item
  \textbf{Development Speed vs.~Execution Speed:} Python offers fast
  development cycles but is slower in execution compared to Rust. Since
  AI processing is not real-time critical, Python was a suitable choice.
\item
  \textbf{Alternative Considered:} Implementing AI processing in Rust,
  but the lack of mature AI libraries made Python the more practical
  choice.
\end{itemize}

\hypertarget{postgresql-for-database-storage}{%
\subsection{4. PostgreSQL for Database
Storage}\label{postgresql-for-database-storage}}

\hypertarget{reasoning-3}{%
\subsubsection{\texorpdfstring{\textbf{Reasoning:}}{Reasoning:}}\label{reasoning-3}}

PostgreSQL was chosen as the primary database due to its robust support
for relational data, indexing, and transaction integrity. It is
well-suited for storing query logs, caching blockchain data, and
managing user interactions.

\hypertarget{tradeoffs-considered-3}{%
\subsubsection{\texorpdfstring{\textbf{Tradeoffs
Considered:}}{Tradeoffs Considered:}}\label{tradeoffs-considered-3}}

\begin{itemize}
\tightlist
\item
  \textbf{Scalability vs.~Simplicity:} NoSQL databases like MongoDB were
  considered but deemed unnecessary since the data model is largely
  structured.
\item
  \textbf{Alternative Considered:} A blockchain-based storage solution
  was also considered but was ruled out due to performance constraints
  and unnecessary complexity for non-on-chain data.
\end{itemize}

\hypertarget{next.js-with-tailwind-for-frontend}{%
\subsection{5. Next.js with Tailwind for
Frontend}\label{next.js-with-tailwind-for-frontend}}

\hypertarget{reasoning-4}{%
\subsubsection{\texorpdfstring{\textbf{Reasoning:}}{Reasoning:}}\label{reasoning-4}}

Next.js was selected for its performance optimizations, server-side
rendering(SSR), and static site generation(SSG) capabilities, which
enhance user experience by reducing load times. Tailwind CSS was chosen
for its utility-first approach, making UI development efficient.

\hypertarget{tradeoffs-considered-4}{%
\subsubsection{\texorpdfstring{\textbf{Tradeoffs
Considered:}}{Tradeoffs Considered:}}\label{tradeoffs-considered-4}}

\begin{itemize}
\tightlist
\item
  \textbf{Performance vs.~Developer Experience:} Next.js provides a good
  balance between frontend speed and developer productivity. Alternative
  frameworks like React with Vite were considered but did not offer as
  many built-in optimizations.
\end{itemize}

To further validate our architectural direction, we employ the Architecture Tradeoff Analysis Method (ATAM). This structured evaluation highlights how our system handles quality attributes like modifiability, performance, and security, while identifying risks and their mitigation strategies.

\begin{center}\rule{0.5\linewidth}{0.5pt}\end{center}

\hypertarget{atamarchitecture-tradeoff-analysis-method}{%
\section{ATAM(Architecture Tradeoff Analysis
Method)}\label{atamarchitecture-tradeoff-analysis-method}}

\hypertarget{business-goals}{%
\subsection{\texorpdfstring{\textbf{1. Business
Goals}}{1. Business Goals}}\label{business-goals}}

The core objectives of SWQuery:

\begin{itemize}
\tightlist
\item
  Efficiently query \textbf{Solana blockchain data} using OpenAI and
  Helius.
\item
  Provide a responsive web interface for users to interact with
  blockchain insights.
\item
  Ensure low-latency responses for users.
\item
  Maintain security and reliability in handling sensitive blockchain
  data.
\end{itemize}

\hypertarget{quality-attributes}{%
\subsection{\texorpdfstring{\textbf{2. Quality
Attributes}}{2. Quality Attributes}}\label{quality-attributes}}

The architecture should be evaluated based on the following
\textbf{quality attributes}:

\begin{longtable}[]{@{}
  >{\raggedright\arraybackslash}p{(\columnwidth - 2\tabcolsep) * \real{0.5000}}
  >{\raggedright\arraybackslash}p{(\columnwidth - 2\tabcolsep) * \real{0.5000}}@{}}
\toprule()
\begin{minipage}[b]{\linewidth}\raggedright
\textbf{Attribute}
\end{minipage} & \begin{minipage}[b]{\linewidth}\raggedright
\textbf{Why It Matters}
\end{minipage} \\
\midrule()
\endhead
\textbf{Performance} & Queries should be processed quickly with minimal
delay. \\
\textbf{Scalability} & The system should handle increased query volume
as user adoption grows. \\
\textbf{Security} & Data integrity and protection against unauthorized
access are crucial. \\
\textbf{Maintainability} & The system should allow easy updates to
backend services and SDK improvements. \\
\textbf{Interoperability} & The architecture should support interactions
between frontend, backend, SDK, and external APIs. \\
\textbf{Reliability} & The system should handle failures gracefully and
ensure uptime. \\
\bottomrule()
\end{longtable}

\hypertarget{architectural-approaches}{%
\subsection{\texorpdfstring{\textbf{3. Architectural
Approaches}}{3. Architectural Approaches}}\label{architectural-approaches}}

Your architecture incorporates several key design choices:

\begin{enumerate}
\def\labelenumi{\arabic{enumi}.}
\tightlist
\item
  \textbf{Rust-based Backend}

  \begin{itemize}
  \tightlist
  \item
    \textbf{Pros}: High performance, memory safety, and low resource
    consumption.
  \item
    \textbf{Cons}: Steeper learning curve and limited ecosystem compared
    to Node.js.
  \item
    \textbf{Tradeoff}: Performance vs.~Development Speed. Rust improves
    execution efficiency but may slow development.
  \end{itemize}
\item
  \textbf{SWQuery SDK for Blockchain Queries}

  \begin{itemize}
  \tightlist
  \item
    \textbf{Pros}: Modular and reusable SDK for querying Solana.
  \item
    \textbf{Cons}: Dependency on third-party services(OpenAI, Helius,
    Pump Portal, CoinGecko).
  \item
    \textbf{Tradeoff}: Reusability vs.~Dependency Management. SDK
    abstraction simplifies query logic but adds an external dependency.
  \end{itemize}
\item
  \textbf{Python API for OpenAI \& Helius}

  \begin{itemize}
  \tightlist
  \item
    \textbf{Pros}: Python is well-suited for AI-based queries and API
    interactions.
  \item
    \textbf{Cons}: Potential latency when interfacing between Rust and
    Python.
  \item
    \textbf{Tradeoff}: AI Capabilities vs.~Latency. Python simplifies AI
    integration but may slow query execution.
  \end{itemize}
\item
  \textbf{PostgreSQL for Storage}

  \begin{itemize}
  \tightlist
  \item
    \textbf{Pros}: Reliable relational database with strong query
    capabilities.
  \item
    \textbf{Cons}: Potential scalability concerns for high-frequency
    queries.
  \item
    \textbf{Tradeoff}: Structured Data vs.~Scalability. SQL databases
    are robust but may require optimizations for high loads.
  \end{itemize}
\end{enumerate}

\hypertarget{architectural-risks}{%
\subsection{\texorpdfstring{\textbf{4. Architectural
Risks}}{4. Architectural Risks}}\label{architectural-risks}}

Potential risks in the current design:

\begin{itemize}
\tightlist
\item
  \textbf{Latency Bottlenecks}: Python API calls to OpenAI and Helius
  may introduce delays.
\item
  \textbf{Scalability of PostgreSQL}: If many users query blockchain
  data frequently, PostgreSQL may become a bottleneck.
\item
  \textbf{Rust Adoption Complexity}: Hiring developers proficient in
  Rust may be harder compared to more common backend languages.
\end{itemize}

\hypertarget{sensitivity-tradeoff-points}{%
\subsection{\texorpdfstring{\textbf{5. Sensitivity \& Tradeoff
Points}}{5. Sensitivity \& Tradeoff Points}}\label{sensitivity-tradeoff-points}}

\hypertarget{rust-for-backend-development}{%
\subsubsection{\texorpdfstring{\textbf{5.1. Rust for Backend
Development}}{5.1. Rust for Backend Development}}\label{rust-for-backend-development}}

\hypertarget{tradeoffs}{%
\paragraph{\texorpdfstring{\textbf{Tradeoffs:}}{Tradeoffs:}}\label{tradeoffs}}

\textbf{Performance \& Control:}

\begin{itemize}
\tightlist
\item
  Rust provides \textbf{high performance} and \textbf{low-level control}
  over processes, which is crucial for a system interacting with
  blockchain.
\item
  Its \textbf{ownership and borrowing model} ensures memory safety
  without garbage collection, avoiding runtime overhead.
\end{itemize}

\textbf{Development Complexity \& Time Cost:}

\begin{itemize}
\tightlist
\item
  Rust has a \textbf{steep learning curve} due to its strict memory
  safety rules, leading to \textbf{longer development time} compared to
  languages like TypeScript or Go.
\item
  Borrowing and ownership concepts require careful design, slowing down
  initial development but reducing runtime errors.
\end{itemize}

\hypertarget{evaluation}{%
\paragraph{\texorpdfstring{\textbf{Evaluation:}}{Evaluation:}}\label{evaluation}}

\begin{itemize}
\tightlist
\item
  If the system requires \textbf{high performance, low latency, and
  safety guarantees}, Rust is a solid choice despite its complexity.
\item
  If \textbf{faster development} were prioritized over performance, a
  language like \textbf{Node.js or Go} might be considered.
\end{itemize}

\hypertarget{openai-for-ai-processing-1}{%
\subsubsection{\texorpdfstring{\textbf{5.2. OpenAI for AI
Processing}}{5.2. OpenAI for AI Processing}}\label{openai-for-ai-processing-1}}

\hypertarget{tradeoffs-1}{%
\paragraph{\texorpdfstring{\textbf{Tradeoffs:}}{Tradeoffs:}}\label{tradeoffs-1}}

\textbf{Advanced AI Capabilities:}

\begin{itemize}
\tightlist
\item
  OpenAI provides \textbf{state-of-the-art models} that would be
  expensive and time-consuming to develop in-house.
\item
  Reduces complexity by leveraging pre-built models instead of training
  and fine-tuning a proprietary AI.
\end{itemize}

\textbf{Financial \& Time Cost:}

\begin{itemize}
\tightlist
\item
  \textbf{API usage costs can be high}, especially as query volume
  increases. Maintaining cost-effective API usage might require caching
  or optimizing queries.
\item
  \textbf{Dependency risk}: OpenAI is a third-party provider, meaning
  potential API changes, rate limits, or pricing adjustments could
  impact operations.
\item
  \textbf{Latency concerns}: Depending on OpenAI's response times,
  real-time applications might experience slight delays.
\end{itemize}

\hypertarget{evaluation-1}{%
\paragraph{\texorpdfstring{\textbf{Evaluation:}}{Evaluation:}}\label{evaluation-1}}

\begin{itemize}
\tightlist
\item
  If \textbf{high-quality AI processing is crucial} and cost is not a
  primary concern, OpenAI is a strong choice.
\item
  If \textbf{cost minimization is critical}, alternative models like
  \textbf{open-source LLMs(e.g., Llama 3, Mistral)} or
  \textbf{self-hosted solutions} could be explored.
\end{itemize}

\hypertarget{rust-server-vs-python-server}{%
\subsubsection{\texorpdfstring{\textbf{5.3. Rust Server vs Python
Server}}{5.3. Rust Server vs Python Server}}\label{rust-server-vs-python-server}}

\hypertarget{tradeoffs-2}{%
\paragraph{\texorpdfstring{\textbf{Tradeoffs:}}{Tradeoffs:}}\label{tradeoffs-2}}

\textbf{Rust Server(Backend)}

\begin{itemize}
\tightlist
\item
  \textbf{High performance}: Rust is compiled and optimized for speed.
\item
  \textbf{Safe concurrency}: Allows efficient parallel processing, which
  is beneficial for blockchain interactions.
\item
  \textbf{Memory safety without garbage collection}, reducing runtime
  errors.
\end{itemize}

\textbf{Rust Server Downsides:}

\begin{itemize}
\tightlist
\item
  \textbf{Longer development time} due to complexity.
\item
  \textbf{Fewer libraries for AI and data processing}, making Python a
  better fit for AI-heavy workloads.
\end{itemize}

\textbf{Python Server(Agent for OpenAI API)}

\begin{itemize}
\tightlist
\item
  \textbf{Rich AI and data science ecosystem}: Python has mature
  libraries(e.g., TensorFlow, PyTorch, NumPy) that make AI integration
  seamless.
\item
  \textbf{Faster prototyping and development}, reducing time-to-market.
\item
  \textbf{Good for external API calls}(e.g., OpenAI, Helius) due to
  simple syntax and vast HTTP libraries.
\end{itemize}

\textbf{Python Server Downsides:}

\begin{itemize}
\tightlist
\item
  \textbf{Slower execution speed} compared to Rust.
\item
  \textbf{Higher memory usage}, which can impact performance under heavy
  loads.
\item
  \textbf{Not ideal for real-time or highly concurrent tasks} due to
  Python's Global Interpreter Lock(GIL).
\end{itemize}

\hypertarget{evaluation-2}{%
\paragraph{\texorpdfstring{\textbf{Evaluation:}}{Evaluation:}}\label{evaluation-2}}

\begin{itemize}
\tightlist
\item
  Using \textbf{Rust for the main backend} ensures performance and
  control over blockchain queries.
\item
  Using \textbf{Python for AI-related tasks} ensures faster development
  and access to better AI libraries.
\item
  If maintaining two different tech stacks is \textbf{too complex}, one
  option is to move AI processing into the Rust backend using
  \textbf{WebAssembly(Wasm)} or integrating Rust-based ML frameworks.
\end{itemize}

\hypertarget{improvements}{%
\subsection{\texorpdfstring{\textbf{6.
Improvements}}{6. Improvements}}\label{improvements}}

\begin{enumerate}
\def\labelenumi{\arabic{enumi}.}
\tightlist
\item
  \textbf{Reducing API Latency}:

  \begin{itemize}
  \tightlist
  \item
    Introduce \textbf{asynchronous processing} in Rust to optimize calls
    to the Python API.
  \item
    Cache responses for repeated blockchain queries to reduce redundant
    API calls.
  \end{itemize}
\item
  \textbf{Enhancing Scalability}:

  \begin{itemize}
  \tightlist
  \item
    Evaluate a \textbf{NoSQL alternative(e.g., MongoDB)} for handling
    blockchain query results.
  \item
    Implement \textbf{Redis caching} for frequently requested data.
  \end{itemize}
\item
  \textbf{Improving Developer Experience}:

  \begin{itemize}
  \tightlist
  \item
    Provide well-documented APIs for easier onboarding of developers.
  \item
    Consider adding a GraphQL layer to improve flexibility in frontend
    queries.
  \end{itemize}
\end{enumerate}

Building on the architectural blueprint, the next section outlines the Functional Requirements of SWQuery. These requirements represent the concrete behaviors the system must implement, directly tying back to user needs and use cases within the DefAI landscape.

\hypertarget{functional-requirements}{%
\section{\texorpdfstring{\textbf{Functional
Requirements}}{Functional Requirements}}\label{functional-requirements}}

\hypertarget{core-functionalities-existing-features}{%
\subsection{\texorpdfstring{\textbf{1. Core Functionalities (Existing
Features)}}{1. Core Functionalities (Existing Features)}}\label{core-functionalities-existing-features}}

\hypertarget{transaction-retrieval-analysis}{%
\subsubsection{\texorpdfstring{\textbf{1.1 Transaction Retrieval \&
Analysis}}{1.1 Transaction Retrieval \& Analysis}}\label{transaction-retrieval-analysis}}

\begin{itemize}
\tightlist
\item
  \textbf{FR1}: The system must retrieve recent transactions associated
  with a given \textbf{transaction address} within a specified day time
  range. (\texttt{getRecentTransactions(address,\ days)})

  \begin{itemize}
  \tightlist
  \item
    This function is referenced in the SWQuery SDK, in
    \href{https://github.com/SWQuery/swquery/blob/main/swquery/src/client.rs}{swquery/src/client.rs}
    in line 542
  \end{itemize}
\item
  \textbf{FR2}: The system must retrieve transaction details using a
  unique transaction signature. (\texttt{getTransaction(signature)})

  \begin{itemize}
  \tightlist
  \item
    This function is referenced in the SWQuery SDK, in
    \href{https://github.com/SWQuery/swquery/blob/main/swquery/src/client.rs}{swquery/src/client.rs}
    in line 642
  \end{itemize}
\end{itemize}

\hypertarget{account-asset-information}{%
\subsubsection{\texorpdfstring{\textbf{1.2 Account \& Asset
Information}}{1.2 Account \& Asset Information}}\label{account-asset-information}}

\begin{itemize}
\tightlist
\item
  \textbf{FR3}: The system must return the current balance of a given
  \textbf{wallet address.} (\texttt{getBalance(address)})

  \begin{itemize}
  \tightlist
  \item
    This function is referenced in the SWQuery SDK, in
    \href{https://github.com/SWQuery/swquery/blob/main/swquery/src/client.rs}{swquery/src/client.rs}
    in line 764
  \end{itemize}
\item
  \textbf{FR4}: The system must retrieve a list of assets (tokens, NFTs)
  owned by a specific \textbf{wallet address.}
  (\texttt{getAssetsByOwner(owner)})

  \begin{itemize}
  \tightlist
  \item
    This function is referenced in the SWQuery SDK, in
    \href{https://github.com/SWQuery/swquery/blob/main/swquery/src/client.rs}{swquery/src/client.rs}
    in line 702
  \end{itemize}
\end{itemize}

\hypertarget{market-insights}{%
\subsubsection{\texorpdfstring{\textbf{1.3 Market
Insights}}{1.3 Market Insights}}\label{market-insights}}

\begin{itemize}
\tightlist
\item
  \textbf{FR5}: The system must provide a list of the 5 trending tokens
  based on transaction volume and activity in the last 24 hours.
  (\texttt{getTrendingTokens()})

  \begin{itemize}
  \tightlist
  \item
    This function is referenced in the SWQuery SDK, in
    \href{https://github.com/SWQuery/swquery/blob/main/swquery/src/client.rs}{swquery/src/client.rs}
    in line 917
  \end{itemize}
\item
  \textbf{FR6}: The system must allow users to search for a token by
  name or description that can match the token knowledge base and
  retrieve relevant details. (\texttt{searchTokenByName(token\_name)})

  \begin{itemize}
  \tightlist
  \item
    This function is referenced in the SWQuery SDK, in
    \href{https://github.com/SWQuery/swquery/blob/main/swquery/src/client.rs}{swquery/src/client.rs}
    in line 1023(not exist in branch main, right now only available at
    development branch, called v3)
  \end{itemize}
\end{itemize}

\hypertarget{subscription-based-updates}{%
\subsubsection{\texorpdfstring{\textbf{1.4 Subscription-Based
Updates}}{1.4 Subscription-Based Updates}}\label{subscription-based-updates}}

\begin{itemize}
\tightlist
\item
  \textbf{FR7}: The system must enable users to subscribe to transaction
  updates involving a specific account(wallet) by its address.
  (\texttt{accountTransactionSubscription(user\_address,\ account\_address)})

  \begin{itemize}
  \tightlist
  \item
    This function is referenced in the SWQuery SDK, in
    \href{https://github.com/SWQuery/swquery/blob/main/swquery/src/client.rs}{swquery/src/client.rs}
    in line 969
  \end{itemize}
\item
  \textbf{FR8}: The system must allow users to subscribe to transactions
  of a specific token based on the token address.
  (\texttt{tokenTransactionSubscription(user\_address,\ token\_address)})

  \begin{itemize}
  \tightlist
  \item
    This function is referenced in the SWQuery SDK, in
    \href{https://github.com/SWQuery/swquery/blob/main/swquery/src/client.rs}{swquery/src/client.rs}
    in line 988
  \end{itemize}
\item
  \textbf{FR9}: The system must provide token notifications when they
  are launched to the market.
  (\texttt{newTokenSubscriptions(user\_address)})

  \begin{itemize}
  \tightlist
  \item
    This function is referenced in the SWQuery SDK, in
    \href{https://github.com/SWQuery/swquery/blob/main/swquery/src/client.rs}{swquery/src/client.rs}
    in line 1007
  \end{itemize}
\end{itemize}

\hypertarget{roadmap-features-future-enhancements}{%
\subsection{\texorpdfstring{\textbf{2. Roadmap Features (Future
Enhancements)}}{2. Roadmap Features (Future Enhancements)}}\label{roadmap-features-future-enhancements}}

\hypertarget{ai-powered-query-analysis}{%
\subsubsection{\texorpdfstring{\textbf{2.1 AI-Powered Query \&
Analysis}}{2.1 AI-Powered Query \& Analysis}}\label{ai-powered-query-analysis}}

\begin{itemize}
\tightlist
\item
  \textbf{FR10}: The system must integrate multiple AI agents (e.g.,
  DeepSeek, Claude, GPT models) to assist with complex queries and
  analysis. (\texttt{Multiple\ Agents\ Integration})
\end{itemize}

\hypertarget{automated-token-nft-operations}{%
\subsubsection{\texorpdfstring{\textbf{2.2 Automated Token \& NFT
Operations}}{2.2 Automated Token \& NFT Operations}}\label{automated-token-nft-operations}}

\begin{itemize}
\tightlist
\item
  \textbf{FR11}: The system must integrate with \textbf{Jupiter} to
  facilitate token swaps and operations using AI-driven automation.
  (\texttt{Jupiter\ Integration})
\item
  \textbf{FR12}: The system must integrate with \textbf{Magic Eden} to
  support NFT operations such as listing, purchasing, and monitoring
  market trends. (\texttt{Magic\ Eden\ Integration})
\end{itemize}

\hypertarget{sentiment-risk-analysis}{%
\subsubsection{\texorpdfstring{\textbf{2.3 Sentiment \& Risk
Analysis}}{2.3 Sentiment \& Risk Analysis}}\label{sentiment-risk-analysis}}

\begin{itemize}
\tightlist
\item
  \textbf{FR13}: The system must integrate with \textbf{Twitter (X)} to
  analyze market sentiment by tracking discussions around specific
  projects or tokens. (\texttt{Twitter\ Integration})
\item
  \textbf{FR14}: The system must provide \textbf{rug pull risk
  assessment} by integrating with \textbf{RugCheck API} to evaluate the
  legitimacy of a project/token. (\texttt{Rug\ Check})
\item
  \textbf{FR15}: The system must analyze \textbf{social media activity}
  of a project (e.g., community engagement, profile name changes, post
  frequency) to assess its credibility. (\texttt{Social\ Media\ Check})
\end{itemize}

Finally, to ensure quality and consistency in implementation, we present Requirements Test Cases. Each test case aligns with a functional requirement, enabling systematic validation of SWQuery’s behavior and ensuring that each component performs as intended under realistic scenarios.

\begin{center}\rule{0.5\linewidth}{0.5pt}\end{center}

\hypertarget{requirements-test-cases}{%
\section{\texorpdfstring{\textbf{Requirements Test
Cases}}{Requirements Test Cases}}\label{requirements-test-cases}}

\hypertarget{core-functionalities-existing-features-1}{%
\subsection{\texorpdfstring{\textbf{1. Core Functionalities (Existing
Features)}}{1. Core Functionalities (Existing Features)}}\label{core-functionalities-existing-features-1}}

\hypertarget{tc1-retrieve-recent-transactions}{%
\subsubsection{\texorpdfstring{\textbf{TC1: Retrieve Recent
Transactions}}{TC1: Retrieve Recent Transactions}}\label{tc1-retrieve-recent-transactions}}

\textbf{Pre-condition:}

\begin{itemize}
\tightlist
\item
  Address: \texttt{0xABC123...}
\item
  Days: \texttt{7}
\item
  The address has at least 3 transactions in the last 7 days.
\end{itemize}

\textbf{Test Execution:}

\begin{enumerate}
\def\labelenumi{\arabic{enumi}.}
\tightlist
\item
  Call \texttt{getRecentTransactions(0xABC123...,\ 7)}.
\item
  The system queries the blockchain for transactions related to the
  address within the specified timeframe.
\end{enumerate}

\textbf{Expected Output:}

\begin{itemize}
\tightlist
\item
  Returns a list of transactions within the last 7 days.
\item
  Each transaction should contain \texttt{signature},
  \texttt{timestamp}, \texttt{amount}, \texttt{token}, and
  \texttt{sender/receiver}.
\end{itemize}

\hypertarget{tc2-retrieve-a-specific-transaction}{%
\subsubsection{\texorpdfstring{\textbf{TC2: Retrieve a Specific
Transaction}}{TC2: Retrieve a Specific Transaction}}\label{tc2-retrieve-a-specific-transaction}}

\textbf{Pre-condition:}

\begin{itemize}
\tightlist
\item
  Transaction signature: \texttt{0xTX123456}
\item
  The transaction exists on the blockchain.
\end{itemize}

\textbf{Test Execution:}

\begin{enumerate}
\def\labelenumi{\arabic{enumi}.}
\tightlist
\item
  Call \texttt{getTransaction(0xTX123456)}.
\item
  The system fetches the transaction details from the blockchain.
\end{enumerate}

\textbf{Expected Output:}

\begin{itemize}
\tightlist
\item
  Returns transaction details, including \texttt{timestamp},
  \texttt{amount}, \texttt{token}, \texttt{from}, \texttt{to}, and
  \texttt{status}.
\end{itemize}

\hypertarget{tc3-retrieve-balance-of-an-address}{%
\subsubsection{\texorpdfstring{\textbf{TC3: Retrieve Balance of an
Address}}{TC3: Retrieve Balance of an Address}}\label{tc3-retrieve-balance-of-an-address}}

\textbf{Pre-condition:}

\begin{itemize}
\tightlist
\item
  Address: \texttt{0xABC123...}
\item
  The address holds \texttt{5.7\ SOL}.
\end{itemize}

\textbf{Test Execution:}

\begin{enumerate}
\def\labelenumi{\arabic{enumi}.}
\tightlist
\item
  Call \texttt{getBalance(0xABC123...)}.
\item
  The system queries the blockchain for the balance.
\end{enumerate}

\textbf{Expected Output:}

\begin{itemize}
\tightlist
\item
  Returns balance: \texttt{5.7\ SOL}.
\end{itemize}

\hypertarget{tc4-retrieve-assets-owned-by-an-address}{%
\subsubsection{\texorpdfstring{\textbf{TC4: Retrieve Assets Owned by an
Address}}{TC4: Retrieve Assets Owned by an Address}}\label{tc4-retrieve-assets-owned-by-an-address}}

\textbf{Pre-condition:}

\begin{itemize}
\tightlist
\item
  Address: \texttt{0xDEF456...}
\item
  The address owns \texttt{2\ NFTs} and \texttt{3\ different\ tokens}.
\end{itemize}

\textbf{Test Execution:}

\begin{enumerate}
\def\labelenumi{\arabic{enumi}.}
\tightlist
\item
  Call \texttt{getAssetsByOwner(0xDEF456...)}.
\item
  The system queries the blockchain for assets owned by the address.
\end{enumerate}

\textbf{Expected Output:}

\begin{itemize}
\tightlist
\item
  Returns a list of assets, including \texttt{NFTs} and \texttt{tokens}.
\end{itemize}

\hypertarget{tc5-retrieve-trending-tokens}{%
\subsubsection{\texorpdfstring{\textbf{TC5: Retrieve Trending
Tokens}}{TC5: Retrieve Trending Tokens}}\label{tc5-retrieve-trending-tokens}}

\textbf{Pre-condition:}

\begin{itemize}
\tightlist
\item
  The system has collected token transaction data.
\item
  Trending tokens are determined based on volume and transaction count.
\end{itemize}

\textbf{Test Execution:}

\begin{enumerate}
\def\labelenumi{\arabic{enumi}.}
\tightlist
\item
  Call \texttt{getTrendingTokens()}.
\item
  The system analyzes and ranks tokens by trading volume and recent
  activity.
\end{enumerate}

\textbf{Expected Output:}

\begin{itemize}
\tightlist
\item
  Returns a list of top \textbf{5 trending tokens} with details
  (\texttt{name}, \texttt{symbol}, \texttt{volume},
  \texttt{price\ change}).
\end{itemize}

\hypertarget{tc6-subscribe-to-account-transactions}{%
\subsubsection{\texorpdfstring{\textbf{TC6: Subscribe to Account
Transactions}}{TC6: Subscribe to Account Transactions}}\label{tc6-subscribe-to-account-transactions}}

\textbf{Pre-condition:}

\begin{itemize}
\tightlist
\item
  User: \texttt{0xUSER789...}
\item
  Target account: \texttt{0xTARGET123...}
\item
  Subscription service is active.
\end{itemize}

\textbf{Test Execution:}

\begin{enumerate}
\def\labelenumi{\arabic{enumi}.}
\tightlist
\item
  Call
  \texttt{accountTransactionSubscription(0xUSER789...,\ 0xTARGET123...)}.
\item
  A transaction occurs for \texttt{0xTARGET123...}.
\item
  The system detects and notifies the user.
\end{enumerate}

\textbf{Expected Output:}

\begin{itemize}
\tightlist
\item
  User \texttt{0xUSER789...} receives a notification when
  \texttt{0xTARGET123...} makes a transaction.
\end{itemize}

\hypertarget{tc7-subscribe-to-token-transactions}{%
\subsubsection{\texorpdfstring{\textbf{TC7: Subscribe to Token
Transactions}}{TC7: Subscribe to Token Transactions}}\label{tc7-subscribe-to-token-transactions}}

\textbf{Pre-condition:}

\begin{itemize}
\tightlist
\item
  User: \texttt{0xUSER789...}
\item
  Token: \texttt{SOL}
\item
  Subscription service is active.
\end{itemize}

\textbf{Test Execution:}

\begin{enumerate}
\def\labelenumi{\arabic{enumi}.}
\tightlist
\item
  Call \texttt{tokenTransactionSubscription(0xUSER789...,\ SOL)}.
\item
  A transaction occurs for \texttt{SOL}.
\item
  The system detects and notifies the user.
\end{enumerate}

\textbf{Expected Output:}

\begin{itemize}
\tightlist
\item
  User receives a notification when a new transaction involving
  \texttt{SOL} happens.
\end{itemize}

\hypertarget{tc8-subscribe-to-new-token-listings}{%
\subsubsection{\texorpdfstring{\textbf{TC8: Subscribe to New Token
Listings}}{TC8: Subscribe to New Token Listings}}\label{tc8-subscribe-to-new-token-listings}}

\textbf{Pre-condition:}

\begin{itemize}
\tightlist
\item
  User: \texttt{0xUSER789...}
\item
  A new token is listed on the blockchain.
\end{itemize}

\textbf{Test Execution:}

\begin{enumerate}
\def\labelenumi{\arabic{enumi}.}
\tightlist
\item
  Call \texttt{newTokenSubscriptions(0xUSER789...)}.
\item
  A new token \texttt{XYZ} is created and listed.
\item
  The system detects and notifies the user.
\end{enumerate}

\textbf{Expected Output:}

\begin{itemize}
\tightlist
\item
  User receives a notification: \textbf{``New Token XYZ has been
  listed!''}
\end{itemize}

\hypertarget{tc9-search-token-by-name}{%
\subsubsection{\texorpdfstring{\textbf{TC9: Search Token by
Name}}{TC9: Search Token by Name}}\label{tc9-search-token-by-name}}

\textbf{Pre-condition:}

\begin{itemize}
\tightlist
\item
  The token \texttt{Solana} exists on the blockchain.
\end{itemize}

\textbf{Test Execution:}

\begin{enumerate}
\def\labelenumi{\arabic{enumi}.}
\tightlist
\item
  Call \texttt{searchTokenByName("Solana")}.
\item
  The system searches for matching tokens.
\end{enumerate}

\textbf{Expected Output:}

\begin{itemize}
\tightlist
\item
  Returns details of the \texttt{Solana\ (SOL)} token.
\end{itemize}

\hypertarget{roadmap-features-future-enhancements-1}{%
\subsection{\texorpdfstring{\textbf{2. Roadmap Features (Future
Enhancements)}}{2. Roadmap Features (Future Enhancements)}}\label{roadmap-features-future-enhancements-1}}

\hypertarget{tc10-multiple-ai-agent-integration}{%
\subsubsection{\texorpdfstring{\textbf{TC10: Multiple AI Agent
Integration}}{TC10: Multiple AI Agent Integration}}\label{tc10-multiple-ai-agent-integration}}

\textbf{Pre-condition:}

\begin{itemize}
\tightlist
\item
  DeepSeek, Claude, and GPT models are available.
\end{itemize}

\textbf{Test Execution:}

\begin{enumerate}
\def\labelenumi{\arabic{enumi}.}
\tightlist
\item
  Call an AI-based query using SWQuery's AI module.
\item
  The system selects the best AI model for the request.
\end{enumerate}

\textbf{Expected Output:}

\begin{itemize}
\tightlist
\item
  Returns the AI-generated response based on query intent.
\end{itemize}

\hypertarget{tc11-jupiter-integration-token-swaps}{%
\subsubsection{\texorpdfstring{\textbf{TC11: Jupiter Integration (Token
Swaps)}}{TC11: Jupiter Integration (Token Swaps)}}\label{tc11-jupiter-integration-token-swaps}}

\textbf{Pre-condition:}

\begin{itemize}
\tightlist
\item
  User has \texttt{5\ SOL}.
\item
  Jupiter swap API is available.
\end{itemize}

\textbf{Test Execution:}

\begin{enumerate}
\def\labelenumi{\arabic{enumi}.}
\tightlist
\item
  Call \texttt{swapToken("SOL",\ "USDC",\ 5)}.
\item
  The system interacts with Jupiter for execution.
\end{enumerate}

\textbf{Expected Output:}

\begin{itemize}
\tightlist
\item
  Transaction confirmation for \texttt{SOL\ →\ USDC} swap.
\end{itemize}

\hypertarget{tc12-magic-eden-integration-nft-trading}{%
\subsubsection{\texorpdfstring{\textbf{TC12: Magic Eden Integration (NFT
Trading)}}{TC12: Magic Eden Integration (NFT Trading)}}\label{tc12-magic-eden-integration-nft-trading}}

\textbf{Pre-condition:}

\begin{itemize}
\tightlist
\item
  User owns an NFT.
\item
  Magic Eden API is available.
\end{itemize}

\textbf{Test Execution:}

\begin{enumerate}
\def\labelenumi{\arabic{enumi}.}
\tightlist
\item
  Call \texttt{listNFT(nft\_id,\ price)}.
\item
  The system lists the NFT on Magic Eden.
\end{enumerate}

\textbf{Expected Output:}

\begin{itemize}
\tightlist
\item
  Confirmation: \textbf{``NFT listed successfully on Magic Eden.''}
\end{itemize}

\hypertarget{tc13-twitter-market-sentiment-analysis}{%
\subsubsection{\texorpdfstring{\textbf{TC13: Twitter Market Sentiment
Analysis}}{TC13: Twitter Market Sentiment Analysis}}\label{tc13-twitter-market-sentiment-analysis}}

\textbf{Pre-condition:}

\begin{itemize}
\tightlist
\item
  Twitter API is accessible.
\end{itemize}

\textbf{Test Execution:}

\begin{enumerate}
\def\labelenumi{\arabic{enumi}.}
\tightlist
\item
  Call \texttt{analyzeMarketSentiment("Solana")}.
\item
  The system scrapes and analyzes tweets.
\end{enumerate}

\textbf{Expected Output:}

\begin{itemize}
\tightlist
\item
  Returns sentiment score (\texttt{Positive}, \texttt{Neutral},
  \texttt{Negative}).
\end{itemize}

\hypertarget{tc14-rug-check-api-integration}{%
\subsubsection{\texorpdfstring{\textbf{TC14: Rug Check API
Integration}}{TC14: Rug Check API Integration}}\label{tc14-rug-check-api-integration}}

\textbf{Pre-condition:}

\begin{itemize}
\tightlist
\item
  A new token \texttt{XYZ} is being analyzed.
\end{itemize}

\textbf{Test Execution:}

\begin{enumerate}
\def\labelenumi{\arabic{enumi}.}
\tightlist
\item
  Call \texttt{checkRugPullRisk("XYZ")}.
\item
  The system queries RugCheck API.
\end{enumerate}

\textbf{Expected Output:}

\begin{itemize}
\tightlist
\item
  Returns \textbf{risk level} (\texttt{Low}, \texttt{Medium},
  \texttt{High}).
\end{itemize}

\hypertarget{tc15-social-media-project-analysis}{%
\subsubsection{\texorpdfstring{\textbf{TC15: Social Media Project
Analysis}}{TC15: Social Media Project Analysis}}\label{tc15-social-media-project-analysis}}

\textbf{Pre-condition:}

\begin{itemize}
\tightlist
\item
  A token project has a social media profile.
\end{itemize}

\textbf{Test Execution:}

\begin{enumerate}
\def\labelenumi{\arabic{enumi}.}
\tightlist
\item
  Call \texttt{analyzeSocialMedia("XYZ")}.
\item
  The system checks engagement, post frequency, and history.
\end{enumerate}

\textbf{Expected Output:}

\begin{itemize}
\tightlist
\item
  Returns a credibility score and flags suspicious activity.
  Segundo \textcite{wang2022}, os algoritmos de consenso... \cite{griffain2025}
\end{itemize}

\section{Economic Impact}

The convergence of decentralized finance (DeFi) and artificial intelligence (AI), often referred to as DefAI, represents a significant paradigm shift in the architecture and operation of blockchain-based financial systems. This synergy introduces intelligent, data-driven automation into domains traditionally governed by human discretion and opaque practices. Within this evolving landscape, SWQuery emerges as a critical infrastructural component, offering a comprehensive suite of tools for extracting, analyzing, and responding to blockchain data in real time. Its integration into the Solana ecosystem contributes not only to technological innovation but also to meaningful economic advancement.

\subsection{Enhancing Market Transparency and Liquidity}

A central economic challenge in decentralized markets is the prevalence of information asymmetry, wherein only technically adept users or entities with privileged data access can make informed decisions. SWQuery directly addresses this issue by democratizing access to key blockchain insights through its developer-friendly SDK. Functions such as \texttt{getTrendingTokens}, \texttt{searchTokenByName}, and \texttt{analyzeRugPullRisk} empower users to identify emergent opportunities and evaluate asset credibility with reduced technical overhead.

This increased accessibility fosters broader participation in token markets, which, in turn, improves market liquidity and efficiency. Moreover, the integration of real-time WebSocket subscriptions for account and token activity allows applications to react dynamically to network events, facilitating the construction of predictive models, algorithmic trading strategies, and automated risk management frameworks. Such capabilities reduce latency in capital deployment and support a more fluid and responsive financial environment.

\subsection{Stimulating Innovation through Composability}

SWQuery also serves as a catalyst for innovation through its composable infrastructure. By abstracting the complexities of on-chain data interaction, it allows developers to build layered applications without duplicating data-fetching logic or indexing infrastructure. In this regard, SWQuery functions as a middleware primitive—enabling the rapid prototyping and deployment of intelligent decentralized applications (dApps), autonomous investment vehicles, and adaptive governance protocols.

This modularity enhances innovation velocity and lowers entry barriers for new projects. As more applications integrate SWQuery's SDK, a recursive feedback loop is established: each additional integration not only validates the utility of SWQuery but also expands its influence as a foundational data layer within the DefAI ecosystem. This recursive composability is a hallmark of economically robust decentralized platforms.

\subsection{Promoting Security and Stability}

In addition to economic expansion, SWQuery contributes to systemic risk mitigation. The \texttt{analyzeRugPullRisk} functionality exemplifies the practical fusion of AI heuristics with blockchain transparency to identify potentially fraudulent or high-risk assets. By offering developers and users a standardized tool for assessing token legitimacy, SWQuery enhances the integrity of market interactions and fosters safer investment environments.

Such preventative capabilities are essential for the long-term stability of decentralized economies, as they reduce susceptibility to manipulative schemes and speculative bubbles. Increased user confidence translates into higher levels of sustainable participation, reinforcing the economic base of the broader ecosystem.

\subsection{Strategic Role in the DefAI Economy}

From a macroeconomic perspective, SWQuery aligns with the long-term objectives of the DefAI movement: to build intelligent, decentralized, and equitable financial systems. As DeFi protocols scale to manage increasingly complex portfolios and governance structures, the role of data-driven insight becomes critical. SWQuery's provision of accurate, real-time, and composable data feeds positions it as a core infrastructural component for future financial autonomy.

In summary, SWQuery’s contribution to the DefAI ecosystem transcends technical utility. It supports economic democratization, stimulates innovation, promotes security, and enhances systemic resilience. By enabling developers and users to navigate and interact with blockchain data more effectively, SWQuery lays the groundwork for a more transparent, efficient, and inclusive decentralized economy.

\section{Conclusion}

The emergence of SWQuery within the DefAI ecosystem marks a pivotal advancement in how developers and users interface with blockchain data. As decentralized finance continues to evolve in complexity and scale, the demand for intelligent, composable, and reliable data infrastructure has never been more critical. This whitepaper has detailed SWQuery’s functional architecture, technical benchmarks, and its operational dynamics through sequence diagrams and architectural decisions, while also situating the project within a broader market context.

By providing an SDK that abstracts the intricacies of interacting with the Solana blockchain, SWQuery significantly reduces the barrier to entry for decentralized application development. Its modular functions---ranging from asset querying and token analytics to real-time WebSocket subscriptions---equip developers with the tools needed to build adaptive and data-driven financial protocols. Moreover, the system’s integration with AI-powered modules such as rug pull risk analysis demonstrates a forward-looking commitment to enhancing both usability and security in DeFi environments.

SWQuery’s value proposition is further reinforced through its economic impact, enabling greater market transparency, fostering composability, and supporting the development of secure and sustainable financial ecosystems. It not only empowers the current wave of developers but also lays the foundation for future innovation within DefAI.

In closing, SWQuery positions itself as more than a utility—it is an enabling framework for a decentralized, intelligent financial future. As adoption grows and integration deepens, SWQuery is poised to become a cornerstone of the DefAI movement, catalyzing the next generation of on-chain intelligence and decentralized finance applications.

\printbibliography

\end{document}
